\documentclass[10pt,fleqn]{article} % Default font size and left-justified equations
\usepackage[%
    pdftitle={SLCI : Modélisation par schéma blocs},
    pdfauthor={Xavier Pessoles}]{hyperref}
    
\input{style/new_style}
\input{style/macros_SII}
\usepackage{style/schemabloc}
\usepackage{multicol}
%\fichetrue
%\fichefalse

%\proftrue
\proffalse

\tdtrue
%\tdfalse

%\courstrue
\coursfalse

\def\discipline{Sciences \\Industrielles de \\ l'Ingénieur}
\def\xxtete{Sciences Industrielles de l'Ingénieur}

\def\classe{PTSI}
\def\xxnumpartie{Partie 2}
\def\xxpartie{Découverte des Systèmes Linéaires Continus et Invariants\\
Analyse, Modélisation, Résolution}

\def\xxnumchapitre{Chapitre 3}
\def\xxchapitre{Modélisation des SLCI par schémas blocs}

\def\xxtitreexo{Système RAMSES}
\def\xxsourceexo{\hspace{.2cm} \textit{D'après ressources de Florestan Mathurin.} \\
\url{http://florestan.mathurin.free.fr}}


\def\xxposongletx{2}
\def\xxposonglettext{1.45}
\def\xxposonglety{20}
\def\xxonglet{Part. 2 -- Ch. 3}

\def\xxactivite{Application 3}
\def\xxauteur{\textsl{Florestan Mathurin}}

\def\xxcompetences{%
\textsl{%
\textbf{Savoirs et compétences :}\\
%\noindent \textbf{Résoudre :} à partir des modèles retenus :
%\begin{itemize}[label=\ding{112},font=\color{ocre}] 
%\item choisir une méthode de résolution analytique, graphique, numérique;
%\item mettre en \oe{}uvre une méthode de résolution.
%\end{itemize}
%\begin{itemize}[label=\ding{112},font=\color{ocre}] 
%\item \textit{Rés -- C1.1 :} Loi entrée sortie géométrique et cinématique -- Fermeture géométrique.
%\end{itemize}
%
%\noindent \textit{Mod2 -- C4.1 :} Représentation par schéma bloc.
}}

\def\xxfigures{\includegraphics[width=.8\textwidth]{images/img1}
}%figues de la page de garde

\def\xxpied{%
Partie 2 -- Découverte des SLCI\\
Ch. 3 : Modélisation par schémas blocs -- \xxactivite%
}


\setcounter{secnumdepth}{5}
%---------------------------------------------------------------------------


\begin{document}
%\chapterimage{png/Fond_Cin}
\input{style/new_pagegarde}
\vspace{10cm}
\pagestyle{fancy}
\thispagestyle{plain}


\def\columnseprulecolor{\color{ocre}}
\setlength{\columnseprule}{0.4pt} 
\ifprof
\else
\begin{multicols}{2}
\fi

\subsection*{Étude du système de régulation du niveau d'eau d'un bassin du système RAMSES}
\subsubsection*{Présentation du système RAMSES}

Après avoir été confrontée à des orages violents ayant entrainés des inondations exceptionnelles  au début des années 80, la ville de bordeaux a décidé de faire de son programme de lutte contre les inondations une priorité.  

Presque trente ans plus tard et après plus d’un milliard d‘euros de travaux réalisés, le système RAMSES est  l’un des systèmes anti-inondations les plus performants au monde.  

%
%\begin{center}
%\includegraphics[width=.47\textwidth]{images/img1}
%\textit{Observation en temps réel du comportement d’un bassin de stockage depuis la tour de contrôle }
%\end{center}
%\end{minipage}

\begin{center}
\begin{tabular}{|p{.22\textwidth}|p{.22\textwidth}|}
\hline
\multicolumn{2}{|l|}{Le système RAMSES, c’est : } \\
\hline
Plus de $2\,052\;km$ de canalisations de diamètre $300 \; mm$ à $4\,500\;mm$ & 49 pluviographes  \\
\hline
82 bassins d’étalement et de stockage offrant une capacité totale de  $2\,544\, 850\;m^3$.
& 
300 limnimètres (équipement permettant l'enregistrement et la transmission de la mesure de la hauteur d'eau, en un point donné, dans un cours d’eau, un barrage, un réservoir…)  \\
 \hline
 61 stations de pompage  d’un débit total de  $133,4 \; m^3/s$ 
 Un réseau d’échange d’informations et un télé-contrôle centralisé &  31 débitmètres, 6 marégraphes \\
 \hline
\end{tabular}
\end{center}

\begin{center}
\includegraphics[width=.48\textwidth]{images/img2}
\end{center}




Grâce à un réseau de tranchées drainantes, l’eau est stockée localement dans différents bassins puis restituée progressivement à faible débit dans le réseau aval (Garonne ou usine de traitement) au moyen d’un ouvrage hydraulique de régulation.  


\begin{center}
\includegraphics[width=.47\textwidth]{images/img3}
\end{center}





On suppose que toutes les conditions initiales sont nulles.

\subparagraph{}
\textit{Appliquer, pour chacun des modèles de connaissance des constituants du système, la
transformation de Laplace. Puis indiquer sa fonction de transfert, et enfin en déduire son
schéma-bloc.}
\ifprof
\begin{corrige}
\end{corrige}
\else
\fi




Le modèle de connaissance du potentiomètre (interface H/M) n'est jamais donné dans les sujets de
concours, il faut donc être capable de le retrouver !

\subparagraph{}
\textit{Donner cette relation entre $h_c (t)$ et $u_c (t)$ qui assure que $\varepsilon (t)$ soit bien une image de l’erreur
du niveau d’eau. En déduire le schéma-bloc correspondant au potentiomètre.}
\ifprof
\begin{corrige}
\end{corrige}
\else
\fi


La relation entre vitesse angulaire $\omega_m(t)$ et position angulaire $\theta_m(t)$ du moteur, n'est aussi jamais donnée
dans les sujets de concours, il faut donc la connaître.

\subparagraph{}
\textit{Donner donc en précisant les unités, cette relation temporelle générale qui lie vitesse et
position. En déduire le schéma-bloc qui passe de $\Omega_m(p)$ à $\Theta_m(p)$.}
\ifprof
\begin{corrige}
\end{corrige}
\else
\fi



\subparagraph{}
\textit{Donner la variable d’entrée et la variable de sortie du système. Puis, représenter le schéma bloc du système entier en précisant le nom des constituants sous les blocs, ainsi que les flux
d’énergie ou d’information entre les blocs.}
\ifprof
\begin{corrige}
\end{corrige}
\else
\fi


\subparagraph{}
\textit{Déterminer les fonctions de transfert $F_1(p)=\dfrac{H(p)}{H_c(p)}$ lorsque $Q_s(p)=0$ et $F_1(p)=\dfrac{H(p)}{Q_s(p)}$ lorsque $H_c(p)=0$.}

\ifprof
\begin{corrige}
\end{corrige}
\else
\fi

\subparagraph{}
\textit{En déduire, à l’aide du théorème de superposition, l’expression de $H(p)$ en fonction de $H_c(p)$ et $Q_s(p)$.}

\ifprof
\begin{corrige}
\end{corrige}
\else
\fi



\ifprof
\else
\end{multicols}

\begin{center}
\begin{tabular}{|p{3cm}|p{7cm}|l|}
\hline
Moteur & 
Il tourne à une vitesse angulaire de $\omega_m(t)$ pour une tension de commande $u_m(t)$. &
$\tau \dfrac{d\omega_m(t)}{dt} + \omega_m(t) = K_m u_m(t)$ \\
\hline
Réducteur & 
Il réduit l'angle de l'axe de rotation du moteur $\theta_m(t)$
en un angle d'ouverture $\theta_v(t)$ de la vanne.&
$\theta_v(t)=r\theta_m(t)$\\
\hline
Vanne & 
Elle délivre un débit $q_e(t)$ pour un angle d'ouverture $\theta_v(t)$. &
$q_e(t)= K_v \cdot \theta_v(t) $ \\
\hline
Réservoir & 
Il est de section constante $S$, et a pour débit d’entrée
$q_e (t)$ et de sortie $q_s (t)$.&
$q_e(t)-q_s(t)=S\cdot\dfrac{dh(t)}{dt}$\\
\hline
Limnimètre (capteur) & 
Il traduit le niveau d'eau $h(t)$ atteint dans le réservoir
en tension $u_{mes}(t)$ , image de ce niveau. &
$u_{mes}(t) = a \cdot h(t)$ \\
\hline
Potentiomètre (Interface H/M) & 
Il traduit la consigne de niveau d'eau $h_c(t)$ souhaité
en tension $u_c(t)$, image de cette consigne. &
\\
\hline 
Régulateur (comparateur + correcteur) & 
Il compare la tension de consigne $u_c(t)$ à la tension
de mesure $u_{mes}(t)$ pour en déduire la tension $\varepsilon(t)$ ,
image de l’erreur, puis corrige (amplifie) cette tension
$\varepsilon (t)$ en une tension de commande du moteur $u_m(t)$.
&
$\varepsilon(t) = u_c(t)-u_{mes}(t)$

et

$u_m(t)=A\varepsilon(t)$\\
\hline
\end{tabular}
\end{center}

$\tau$, $K_m$, $r$, $K_v$, $S$, $a$ et $A$ sont des coefficients constants. 
\fi

\end{document}


