\documentclass[10pt,fleqn]{article} % Default font size and left-justified equations
\usepackage[%
    pdftitle={SLCI : Modélisation par schéma blocs},
    pdfauthor={Xavier Pessoles}]{hyperref}
    
\input{style/new_style}
\input{style/macros_SII}
\usepackage{style/schemabloc}
\usepackage{multicol}
%\fichetrue
%\fichefalse

%\proftrue
\proffalse

\tdtrue
%\tdfalse

%\courstrue
\coursfalse

\def\discipline{Sciences \\Industrielles de \\ l'Ingénieur}
\def\xxtete{Sciences Industrielles de l'Ingénieur}

\def\classe{PTSI}
\def\xxnumpartie{Partie 2}
\def\xxpartie{Découverte des Systèmes Linéaires Continus et Invariants\\
Analyse, Modélisation, Résolution}

\def\xxnumchapitre{Chapitre 3}
\def\xxchapitre{Modélisation des SLCI par schémas blocs}

\def\xxtitreexo{Asservissement d'un axe de robot cartésien}
\def\xxsourceexo{\hspace{.2cm} D'après concours externe de l'agrégation de Génie Mécanique -- Épreuve d'Automatique et d'Informatique Industrielle  -- 2008.}


\def\xxposongletx{2}
\def\xxposonglettext{1.45}
\def\xxposonglety{20}
\def\xxonglet{Part. 2 -- Ch. 3}

\def\xxactivite{TD 3}
\def\xxauteur{\textsl{Xavier Pessoles}}

\def\xxcompetences{%
\textsl{%
\textbf{Savoirs et compétences :}\\
%\noindent \textbf{Résoudre :} à partir des modèles retenus :
%\begin{itemize}[label=\ding{112},font=\color{ocre}] 
%\item choisir une méthode de résolution analytique, graphique, numérique;
%\item mettre en \oe{}uvre une méthode de résolution.
%\end{itemize}
%\begin{itemize}[label=\ding{112},font=\color{ocre}] 
%\item \textit{Rés -- C1.1 :} Loi entrée sortie géométrique et cinématique -- Fermeture géométrique.
%\end{itemize}
%
%\noindent \textit{Mod2 -- C4.1 :} Représentation par schéma bloc.
}}

\def\xxfigures{\includegraphics[width=.8\textwidth]{images/structure}
}%figues de la page de garde

\def\xxpied{%
Partie 2 -- Découverte des SLCI\\
Ch. 3 : Modélisation par schémas blocs -- \xxactivite%
}


\setcounter{secnumdepth}{5}
%---------------------------------------------------------------------------


\begin{document}
%\chapterimage{png/Fond_Cin}
\input{style/new_pagegarde}
\vspace{10cm}
\pagestyle{fancy}
\thispagestyle{plain}


\def\columnseprulecolor{\color{ocre}}
\setlength{\columnseprule}{0.4pt} 
\ifprof
\else
\begin{multicols}{2}
\fi

\begin{obj} 
TODO
\end{obj}

\subsection*{Mise en situation}

\begin{center}
\includegraphics[width=.45\textwidth]{images/RobotCartesien}
\end{center}


\begin{center}

\includegraphics[width=.45\textwidth]{images/chaine_01}

\includegraphics[width=.45\textwidth]{images/chaine_02}
\end{center}

Une ligne d'assemblage est équipée d'un robot cartésien permettant de réaliser des opérations de chargement et de déchargement. Ce robot possède 3 axes perpendiculaires entre eux. 
Nous nous intéressons à la commande d'un des axes dont ont donne un premier modèle.

On note : 
\begin{itemize}[label=$\blacksquare$,font=\tiny]
\item $\omega_m(t)$, $\omega_p(t)$ ($\text{rad}\cdot \text{s}^{-1}$) : fréquences de rotation du moteur et de la poulie;
\item $Y_c(t)$, $Y_m(t)$ ($\text{m}$): position de la courroie et du chariot;
\item $V_c(t)$, $V_m(t)$ ($\text{m}\cdot \text{s}^{-1}$): vitesses linéaires de déplacement de la courroie et du chariot;
\item $R =45\cdot 10^{-3} \; \text{m}$ : rayon de la poulie;
\item $n=\dfrac{\omega_m}{\omega_p} = 7,25$ : rapport de transmission du réducteur simple;
\item $K_e = 300\, 000 \; \text{N}\cdot \text{m}^{-1} $: raideur équivalent de la courroie;
\item $f_m = 0,005\; \text{N}\cdot\text{m}\cdot\text{rad}^{-1}\cdot\text{s}^{-1}$ et $f_g = 100\; \text{N}\cdot\text{m}^{-1}\cdot\text{s}^{-1}$ : frottement visqueux équivalents dans les éléments en rotation et dans la glissière;
\item $M = 150 \; \text{kg}$ : masse du chariot et de la charge à déplacer;
\item $J_e = 0,0062 \; \text{kg} \cdot \text{m}^2 $ : inertie équivalente de l'arbre de la poulie et de l'arbre moteur ramenée à l'arbre moteur.
\end{itemize}


\subsection*{Modélisation du système mécanique}
La relation entre la fréquence de rotation du moteur et la vitesse de déplacement de la courroie est donnée par la relation suivante :
\begin{equation}
V_c(t) = R \cdot  \omega_p(t) = \dfrac{R}{n} \cdot \omega_m(t)
\end{equation}



L'effort dû à l'élasticité de la courroie s'opposant à l'effort de traction de la courroie, on a :
\begin{equation}
F_c(t) = K_e \left(Y_c(t) - Y_m(t) \right)
\end{equation}

L'application du théorème de la résultante dynamique au chariot en projection sur l'axe de déplacement du chariot on a : 
\begin{equation}
M\dfrac{\text{d}V_m(t)}{\text{d}t} = F_c(t) - f_g V_m(t)
\end{equation}

L'application du théorème de l'énergie cinétique à l'ensemble \{Moteur, Réducteur, Courroie\} :
\begin{equation}
\dfrac{\text{d}}{\text{d}t}\left[ \dfrac{1}{2} J_e \omega_m(t)^2 \right] =  C_m(t) \omega_m(t) - f_m\omega_m(t)^2 - F_c(t) V_c(t) 
\end{equation}

\subparagraph{} \textit{Donner dans le domaine temporel puis dans le domaine de Laplace la relation entre $V_m(t)$ et $Y_m(t)$. Comment traduire cette relation en schéma bloc ?}
\ifprof
\begin{corrige}
La vitesse étant la dérivée de la position, on a $V_m(t) = \dfrac{\text{d}Y_m(t)}{\text{d}t}$. Dans le domaine de Laplace sous réserve que les conditions initiales soient nulles, on a 
$V_m(p)=pY_m(p)$. En conséquence, 

\begin{center}
\begin{tikzpicture}
\sbEntree{E}

\sbBloc[4]{blocA}{$\dfrac{1}{p}$}{E}
    \sbRelier[$V_m(p)$]{E}{blocA}

\sbSortie{S}{blocA}
    \sbRelier{blocA}{S}
    \sbNomLien[0.8]{S}{$Y_m(p)$}
   
\end{tikzpicture}
\end{center}

\end{corrige}
\else
\fi

\subparagraph{} \textit{Appliquer la transformée de Laplace à chacune des équations et établir la modélisation par schéma bloc associée à chacune d'entre elles.}
\ifprof
\begin{corrige}
On a :
\begin{itemize}
\item [\tiny\color{violet}{ $\blacksquare$}] $V_c(p)=R\Omega_p(p)=\dfrac{R}{n}\cdot \Omega_m(p)$;
\item [\tiny\color{violet}{ $\blacksquare$}] $F_c(p)=K_e\left(Y_c(p) - Y_m(p) \right)$;
\item [\tiny\color{violet}{ $\blacksquare$}] $MpV_m(p) = F_c(p) - f_g V_m(p)$

 $\Longleftrightarrow V_m(p)\left(Mp + f_g \right) = F_c(p) $.
\end{itemize}

L'équation provenant du théorème de l'énergie cinétique peut s'exprimer de la façon suivante en calculant la dérivée et en remplaçant $V_C$ : 
\begin{eqnarray*}J_e \dot{\omega}_m(t) \omega_m(t)= 
C_m(t) \omega_m(t) - f_m\omega_m(t)^2 \\
 - F_c(t) \dfrac{R}{n} \cdot \omega_m(t) 
\end{eqnarray*}
$$
J_e \dot{\omega}_m(t) =  
C_m(t) - f_m\omega_m(t) - F_c(t) \dfrac{R}{n} 
$$
D'où  $J_e p\Omega_m(p) =  
C_m(p) - f_m\Omega_m(p) - F_c(p) \dfrac{R}{n} $. On a donc :
$\Omega_m(p) \left( J_e p + f_m\right) =  
C_m(p)  - F_c(p) \dfrac{R}{n} $ 

\end{corrige}
\begin{corrige}

On peut alors établir les schémas blocs suivants :

\begin{center}
\begin{tikzpicture}
\sbEntree{E}

\sbBloc[4]{blocA}{$\dfrac{R}{n}$}{E}
    \sbRelier[$\Omega_m(p)$]{E}{blocA}

\sbSortie{S}{blocA}
    \sbRelier{blocA}{S}
    \sbNomLien[0.8]{S}{$V_c(p)$}
\end{tikzpicture}
\end{center}


\begin{center}
\begin{tikzpicture}
\sbEntree{E}

\sbComp{c1}{E}
    \sbRelier[$Y_c(p)$]{E}{c1}
    
\sbBloc[7]{b1}{$K_e$}{c1}
    \sbRelier[$Y_c(p)-Y_m(p)$]{c1}{b1}

\sbSortie{S}{b1}
    \sbRelier{b1}{S}
    \sbNomLien[0.8]{S}{$F_c(p)$}
    
\sbDecaleNoeudy[4]{c1}{n1}
    \sbRelier[$Y_m(p)$]{n1}{c1}

\end{tikzpicture}
\end{center}

Ce schéma est équivalent à 
\begin{center}
\begin{tikzpicture}
\sbEntree{E}

\sbComp{c1}{E}
    \sbRelier[$V_c(p)$]{E}{c1}
    
\sbBloc[7]{b1}{$\dfrac{K_e}{p}$}{c1}
    \sbRelier[$V_c(p)-V_m(p)$]{c1}{b1}

\sbSortie{S}{b1}
    \sbRelier{b1}{S}
    \sbNomLien[0.8]{S}{$F_c(p)$}
    
\sbDecaleNoeudy[4]{c1}{n1}
    \sbRelier[$V_m(p)$]{n1}{c1}

\end{tikzpicture}
\end{center}

\begin{center}
\begin{tikzpicture}
\sbEntree{E}

\sbBloc[4]{blocA}{$\dfrac{1}{Mp+f_g}$}{E}
    \sbRelier[$F_c(p)$]{E}{blocA}

\sbSortie{S}{blocA}
    \sbRelier{blocA}{S}
    \sbNomLien[0.8]{S}{$V_m(p)$}
\end{tikzpicture}
\end{center}

\begin{center}
\begin{tikzpicture}
\sbEntree{E}

\sbComp{c1}{E}
    \sbRelier[$C_m(p)$]{E}{c1}
    
\sbBloc{b1}{$\dfrac{1}{J_ep + f_m}$}{c1}
    \sbRelier{c1}{b1}

\sbSortie{S}{b1}
    \sbRelier{b1}{S}
    \sbNomLien[0.8]{S}{$\Omega_m(p)$}
    
\sbDecaleNoeudy[4]{S}{n1}
\sbBlocr[3]{b2}{$\dfrac{R}{n}$}{n1}
    \sbRelier{n1}{b2}
    \sbNomLien[0.8]{n1}{$F_c(p)$}
    \sbRelierxy{b2}{c1}

\end{tikzpicture}
\end{center}
\end{corrige}
\else
\fi
\subparagraph{} \textit{En utilisant la figure \ref{fig1}, établir le schéma bloc associé au déplacement du chariot en fonction du couple fourni par le moteur. On explicitera chacune des fonctions de transfert $H_i$.}

\subparagraph{} \textit{Après avoir modifié le schéma bloc, exprimer la fonction de transfert $H=\dfrac{C_m}{V_M}$ en fonction des différentes fonctions de transfert $H_i$.}



\subsection*{Modélisation de la commande du système}

\ifprof
\else
\end{multicols}
\fi

\ifprof

\begin{figure*}[h]
\begin{center}
\begin{tikzpicture}
% Blocs de la chaine directe
\sbEntree{E}
\sbComp{compA}{E}
    \sbRelier[$C_m$]{E}{compA}

\sbBloc{blocA}{$\dfrac{1}{f_m+J_e p}$}{compA}
    \sbRelier{compA}{blocA}
    
\sbBloc{blocB}{$\dfrac{R}{n}$}{blocA}
    \sbRelier[$\omega_m$]{blocA}{blocB}

\sbComph{compB}{blocB}
    \sbRelier[$V_c$]{blocB}{compB}
    
\sbBloc{blocC}{$\dfrac{K_e}{p}$}{compB}
    \sbRelier{compB}{blocC}    

\sbBloc{blocD}{$\dfrac{1}{f_g+Mp}$}{blocC}
    \sbRelier[$F_c$]{blocC}{blocD}

\sbSortie{S}{blocD}
    \sbRelier{blocD}{S}
    \sbNomLien[0.8]{S}{$V_M$}

% Chaine de retour
\sbRenvoi[-3]{blocD-S}{compB}{}

\sbDecaleNoeudy{blocB}{n1}
\sbBlocr{r1}{$\dfrac{R}{n}$}{n1}
    \sbRelierxy{r1}{compA}
    \sbRelieryx{blocC-blocD}{r1}
   
\end{tikzpicture}
\end{center}
\caption{Schéma bloc \label{fig1}}
\end{figure*}


\begin{figure*}[h]
\begin{center}
\begin{tikzpicture}
% Blocs de la chaine directe
\sbEntree{E}
\sbComp{compA}{E}
    \sbRelier[$C_m$]{E}{compA}

\sbBloc{blocA}{$\dfrac{1}{f_m+J_e p}$}{compA}
    \sbRelier{compA}{blocA}
    
\sbBloc{blocB}{$\dfrac{R}{n}$}{blocA}
    \sbRelier[$\omega_m$]{blocA}{blocB}

\sbComp{compB}{blocB}
    \sbRelier[$V_c$]{blocB}{compB}
    
\sbBloc{blocC}{$\dfrac{K_e}{p}$}{compB}
    \sbRelier{compB}{blocC}    

\sbBloc{blocD}{$\dfrac{1}{f_g+Mp}$}{blocC}
    \sbRelier[$F_c$]{blocC}{blocD}

\sbSortie[4]{S}{blocD}
    \sbRelier{blocD}{S}
    \sbNomLien[0.8]{S}{$V_M$}

% Chaine de retour
\sbRenvoi{blocD-S}{compB}{}

\sbDecaleNoeudy{compB}{n1}
\sbDecaleNoeudx[-1]{S}{n2}
\sbBlocr{r1}{$f_g+Mp$}{n1}
\sbBlocr{r2}{$\dfrac{R}{n}$}{r1}
   \sbRelieryx{n2}{r1}
   \sbRelier{r1}{r2}
    \sbRelierxy{r2}{compA}
    
   
\end{tikzpicture}
\end{center}
\caption{Schéma bloc transformé \label{fig2}}
\end{figure*}

\else
 
\begin{figure*}[h]
\begin{center}
\begin{tikzpicture}
% Blocs de la chaine directe
\sbEntree{E}
\sbComp{compA}{E}
    \sbRelier[$C_m$]{E}{compA}

\sbBloc{blocA}{$H_1(p)$}{compA}
    \sbRelier{compA}{blocA}
    
\sbBloc{blocB}{$H_2(p)$}{blocA}
    \sbRelier{blocA}{blocB}

\sbComph{compB}{blocB}
    \sbRelier{blocB}{compB}
    
\sbBloc{blocC}{$H_3(p)$}{compB}
    \sbRelier{compB}{blocC}    

\sbBloc{blocD}{$H_4(p)$}{blocC}
    \sbRelier{blocC}{blocD}

\sbSortie{S}{blocD}
    \sbRelier{blocD}{S}
    \sbNomLien[0.8]{S}{$V_M$}

% Chaine de retour
\sbRenvoi[-3]{blocD-S}{compB}{}

\sbDecaleNoeudy{blocB}{n1}
\sbBlocr{r1}{$H_5(p)$}{n1}
    \sbRelierxy{r1}{compA}
    \sbRelieryx{blocC-blocD}{r1}
   
\end{tikzpicture}
\end{center}
\caption{Schéma bloc à remplir \label{fig1}}
\end{figure*}
\fi


\end{document}


