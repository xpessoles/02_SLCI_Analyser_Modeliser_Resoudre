\documentclass[10pt]{article}
\input{style/coursHeadings}
\input{style/programHeadings}
\input{style/macros_SII}
\input{style/macros_Titres}
\input{style/macros_Frames}

%Si le boolen xp est vrai : compilation pour xabi
%Sinon compilation Damien
\newboolean{xp}
\setboolean{xp}{true}

\newboolean{prof}
\setboolean{prof}{true}

\newif\ifprof
%\proftrue
\proffalse

\newboolean{td}
\setboolean{td}{true}

\usepackage[%
    pdftitle={},
    pdfauthor={Florestan Mathurin},
    colorlinks=true,
    linkcolor=blue,
    citecolor=magenta]{hyperref}

\def\discipline{Sciences Industrielles de l'Ingénieur}

\def\xxtitre{\ifthenelse{\boolean{xp}}{CI 2 -- SLCI : Étude du comportement des Systèmes Linéaires Continus Invariants}{}}

\def\xxsoustitre{\ifthenelse{\boolean{xp}}{
Chapitre 3 -- Modélisation des Systèmes Linéaires Continus Invariants \\ Modélisation par schémas blocs}{
}}


\def\xxauteur{\ifthenelse{\boolean{xp}}{
\noindent Florestan \textsc{Mathurin}}{
}}


\def\xxpied{\ifthenelse{\boolean{xp}}{
CI 2 : SLCI \\
Ch 3 : Modélisation des SLCI par schéma bloc -- Application 2 --\ifthenelse{\boolean{prof}}{P}{E}%
}{
}}

\def\xxcathegorie{\ifthenelse{\boolean{xp}}{
2013 -- 2014 \\
Xavier \textsc{Pessoles}}{
Informatique - Cours}}

%---------------------------------------------------------------------------


\begin{document}

\ifthenelse{\boolean{xp}}{\input{style/enteteXP}}{\input{style/enteteDI}}



%\renewcommand{\baselinestretch}{1.2}
%\setlength{\parskip}{2ex plus 0.5ex minus 0.2ex}



%\begin{comp}
%\noindent \textbf{Résoudre :} à partir des modèles retenus :
%\begin{itemize}
%\item choisir une méthode de résolution analytique, graphique, numérique;
%\item mettre en \oe{}uvre une méthode de résolution.
%\end{itemize}

%\noindent \textit{Rés -- C1.1 :} Loi entrée sortie géométrique et cinématique -- Fermeture géométrique.

%\noindent \textit{Mod2 -- C4.1 :} Représentation par schéma bloc.
%\end{comp}


\setcounter{subparagraph}{0}

\begin{flushright}
\textit{D'après ressources de Florestan Mathurin.}

\url{http://florestan.mathurin.free.fr/}
\end{flushright}

\subsection*{Modélisation d'une enceinte chauffante}

\begin{minipage}[c]{.47\linewidth}

Le système représenté ci contre est chargé de maintenir la température d’une enceinte. Le chauffage est assuré par un échangeur thermique. Une vanne permet de réguler le débit dans l’échangeur. 

On note $\alpha(t)$ l’angle d’ouverture de la vanne, $q(t)$ le débit dans l’échangeur, $\theta_1 (t)$ la 
température en sortie de l’échangeur, $\theta(t)$ la température de l’enceinte.
\end{minipage}\hfill
\begin{minipage}[c]{.47\linewidth}
\begin{center}
\includegraphics[width=.95\textwidth]{images/img1}
\end{center}
\end{minipage}

On donne les modèles de connaissance qui régissent le système : 
\begin{itemize}
\item $q(t)=k_0 \alpha(t)$ (loi de fonctionnement de la vanne donnant le débit en fonction de l’angle d’ouverture de la vanne);
\item $\theta_1(t) + \tau_1 \dfrac{d\theta_1}{dt} = k_1 q(t)$ (loi de transfert de chaleur dans l’échangeur);
\item  $\theta(t) + \tau_2 \dfrac{d\theta}{dt} = k_2 \theta_1(t)$ (loi de transfert de chaleur dans l’enceinte).  
\end{itemize}

On suppose que toutes les conditions initiales sont nulles. L’entrée du système est l’angle d’ouverture de la vanne $\alpha(t)$ et la sortie, la température de l’enceinte $\theta(t)$. 

\subparagraph{}
\textit{Traduire dans le domaine de Laplace les équations du modèle de connaissance. En déduire les différents modèles de comportement et les fonctions de transfert associées.}

\subparagraph{}
\textit{Représenter le système par un schéma-bloc faisant intervenir les 3 blocs précédemment définis.}


Afin de réguler la température, on choisit de motoriser la vanne. On installe un capteur dans l’enceinte qui permet de mesurer la température et la de traduire en une tension $u_{mes} (t)$ (on peut modéliser le capteur par un gain pur $K_{mes} =0,02$). La tension $u_{mes} (t)$ est comparée à la tension de consigne $u_c(t)$ issue d’un transducteur de fonction de transfert $T(p)$. En fonction de cet écart amplifié par un correcteur de gain $K_c$ , la vanne s’ouvre ou se ferme. Le schéma ci-dessous précise l’architecture du système. 

\begin{center}
\includegraphics[width=.6\textwidth]{images/img2}
\end{center}

On donne la fonction de transfert du moteur qui est $M(p)=\dfrac{\alpha(p)}{U_m(p)}=\dfrac{K}{1+\tau p}$.

\subparagraph{}
\textit{Représenter par un schéma-bloc le système régulé dont l’entrée est la température $\theta_c (p)$. }


\subparagraph{}
\textit{Quelle doit être la fonction de transfert du transducteur de façon à annuler l’écart $\varepsilon(p)$ quand la  température de consigne et la température de l’enceinte sont égales ? }




\end{document}

