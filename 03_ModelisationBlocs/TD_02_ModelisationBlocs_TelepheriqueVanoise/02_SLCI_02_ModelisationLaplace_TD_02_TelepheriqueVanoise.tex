\documentclass[10pt]{article}
\input{style/coursHeadings}
\input{style/programHeadings}
\input{style/macros_SII}
\input{style/macros_Titres}
\input{style/macros_Frames}

%Si le boolen xp est vrai : compilation pour xabi
%Sinon compilation Damien
\newboolean{xp}
\setboolean{xp}{true}

\newboolean{prof}
\setboolean{prof}{true}

\newif\ifprof
%\proftrue
\proffalse

\newboolean{td}
\setboolean{td}{true}

\usepackage[%
    pdftitle={},
    pdfauthor={Xavier Pessoles},
    colorlinks=true,
    linkcolor=blue,
    citecolor=magenta]{hyperref}

\def\discipline{Sciences Industrielles de l'Ingénieur}

\def\xxtitre{\ifthenelse{\boolean{xp}}{Partie 2 -- SLCI : Étude du comportement cinématique des systèmes}{}}

\def\xxsoustitre{\ifthenelse{\boolean{xp}}{
Chapitre 4 -- Étude des chaînes fermées : Détermination des lois Entrée -- Sortie}{
}}


\def\xxauteur{\ifthenelse{\boolean{xp}}{
\noindent Xavier \textsc{Pessoles}}{
}}


\def\xxpied{\ifthenelse{\boolean{xp}}{
CI 4 : Cinématique \\
Ch 4 : Chaînes fermées -- Lois entrée -- sortie -- TD -- \ifthenelse{\boolean{prof}}{P}{E}%
}{
}}

\def\xxcathegorie{\ifthenelse{\boolean{xp}}{
2013 -- 2014 \\
Xavier \textsc{Pessoles}}{
Informatique - Cours}}

%---------------------------------------------------------------------------


\begin{document}

\ifthenelse{\boolean{xp}}{\input{style/enteteXP}}{\input{style/enteteDI}}



%\renewcommand{\baselinestretch}{1.2}
%\setlength{\parskip}{2ex plus 0.5ex minus 0.2ex}



\begin{comp}
\noindent \textbf{Résoudre :} à partir des modèles retenus :
%\begin{itemize}
%\item choisir une méthode de résolution analytique, graphique, numérique;
%\item mettre en \oe{}uvre une méthode de résolution.
%\end{itemize}

%\noindent \textit{Rés -- C1.1 :} Loi entrée sortie géométrique et cinématique -- Fermeture géométrique.

%\noindent \textit{Mod2 -- C4.1 :} Représentation par schéma bloc.
\end{comp}

\section*{Étude du téléphérique Vanoise Express}

\begin{flushright}
\textit{D'après concours E3A -- PSI -- 2014.}
\end{flushright}



Notations :
On notera F(p) la transformée de Laplace d’une fonction du temps f(t).

On note : 
	F(p) la transformée de Laplace d’une fonction du temps f(t) ;
	u(t) : la tension d’alimentation des moteurs ;
	i(t) : l’intensité traversant un moteur ;
	e(t) : la force contre électromotrice d’un moteur ;
	\omega_m (t) : vitesse de rotation d’un moteur ;
	c_m (t) : couple d’un seul moteur ;
	c_r (t) : couple de perturbation engendré par le poids du téléphérique dans une pente et par l’action du vent, ramené sur l’axe des moteurs.	 
Modélisation électrique du MCC

Hypothèses et données :
On suppose les conditions initiales nulles. Les deux moteurs fonctionnent de manière parfaitement identique.
	L=0,59 mH, inductance d’un moteur ;
	R=0,0386 Ω : résistance interne d’un moteur ;
	f=6 N.m.s/rad : coefficient de frottement visqueux équivalent ramené sur l’axe des moteurs ;
	J=800 kg.m² Moment d’inertie total des pièces en rotation, ramené sur l’axe des moteurs ;

Le fonctionnement électromécanique des deux moteurs est décrit par les équations suivantes :
	c_m (t)=k_T⋅i(t) avec k_T=5,67 Nm/A (constante de couple d’un moteur) ;
	e(t)=k_E \omega_m (t) avec k_E=5,77 Vs/rad (constante électrique d’un moteur).

L’application des théorèmes de la dynamique permet d’écrire que : 
	2⋅c_m (t)-c_r (t)=J\omega ̇_m (t)+f⋅\omega_m (t)

L’application de la loi des mailles dans le schéma électrique se traduit par l’équation suivante :
	u(t)=e(t)+Ri(t)+L (di(t))/dt


\end{document}

