\documentclass[11pt,oneside]{article}
\input{style/coursHeadings}
\input{style/programHeadings}


%Si le boolen xp est vrai : compilation pour xabi
%Sinon compilation Damien
\newboolean{xp}
\setboolean{xp}{true}

\newboolean{prof}
\setboolean{prof}{true}

\def\xxtitre{\ifthenelse{\boolean{xp}}{
CI 2 -- SLCI : Etude du comportement des Systèmes Linéaires Continus Invariants}{
Chapitre 1 -- Le système informatique}}

\def\xxsoustitre{\ifthenelse{\boolean{xp}}{
Chapitre 2 -- Modélisation des Systèmes Linéaires Continus Invariants \\ Transformée de Laplace \\
Décomposition en éléments simples}{
}}

\def\xxauteur{\ifthenelse{\boolean{xp}}{
\noindent 2013 -- 2014 \\
Xavier \textsc{Pessoles}}{
Damien Iceta}}


\def\xxpied{\ifthenelse{\boolean{xp}}{
CI 2 : SLCI -- Cours \\
Ch 2 : Modélisation des SLCI -- \ifthenelse{\boolean{prof}}{P}{E}%
}{
TD -- CI 3 : Ingénierie numérique et simulation}}



\ifthenelse{\boolean{xp}}{
\usepackage[%
    pdftitle={SLCI - Modélisation des SLCI},
    pdfauthor={Xavier Pessoles},
    colorlinks=true,
    linkcolor=blue,
    citecolor=magenta]{hyperref}}{
\usepackage[%
    pdftitle={OS et Environnement de développement},
    pdfauthor={Damien Iceta},
    colorlinks=true,
    linkcolor=blue,
    citecolor=magenta]{hyperref}}



\usepackage{pifont}
\sloppy
\hyphenpenalty 10000

\begin{document}


\input{style/entete1}

\begin{center}
 \huge\textsc{\xxtitre}
\end{center}

\begin{center}
 \LARGE\textsc{\xxsoustitre}
\end{center}




\vspace{.5cm}


\section*{Préliminaire}

Le but de ce document n'est pas de se substituer à un cours de mathématiques mais de fournir quelques techniques de calcul pour décomposer une fraction rationnelle en éléments simples. 

\begin{rem}
\begin{enumerate}
\item Dans les fonctions que nous utiliserons, le degré du numérateur sera toujours inférieur ou égal au degré du numérateur : $N(p)\leq D(p)$.
\item On appelle \textbf{pôles} d'une fonction rationnelle les racines du dénominateur. On appelle \textbf{zéros} d'une fonction rationnelle les racines du numérateur.
\item Le numérateur sera un produit de polynômes de degrés inférieurs ou égaux à 2. 
\end{enumerate}
\end{rem}

\section{Cas d'une fonction avec des pôles réels simples -- $deg(N(p)) < deg(D(p))$}

\begin{resultat}
Soit la fonction rationnelle suivante :
$$
F(p) = \dfrac{N(p)}{D(p)} =\dfrac{N(p)}{(p-p_1)\cdot (p-p_2)\cdot \cdot \cdot (p-p_n)}
$$
avec $deg(N(p)) < deg(D(p))$, $p_1\neq p_2\neq p_n$. On peut décomposer la fonction ainsi :
$$
F(p) = \dfrac{\alpha_1}{p-p_1} + \dfrac{\alpha_2}{p-p_2} 
+ \cdot \cdot \cdot + 
\dfrac{\alpha_n}{p-p_n} 
$$
avec $\alpha_i\in \mathbb{R}$.
\end{resultat}

\begin{methode}
\begin{enumerate}
\item On multiplie les deux formes de $F(p)$ par $(p-p_i)$.
\item On pose $p=p_i$.
\item On détermine $\alpha_i$.
\item On réalise l'opération $n$ fois (avec $n=deg(D(p))$.
\end{enumerate}
\end{methode}

\begin{exemple}
\textit{Décomposer la fonction suivante en éléments simples :}
$$
F(p)=\dfrac{p^3+4}{(p+1)(p+2)(p+3)(p+4)}
$$
\end{exemple}

\section{Cas d'une fonction avec des pôles réels simples -- $deg(N(p)) = deg(D(p))$}

\begin{resultat}
Soit la fonction rationnelle suivante :
$$
F(p) = \dfrac{N(p)}{D(p)} =\dfrac{N(p)}{(p-p_1)\cdot (p-p_2)\cdot \cdot \cdot (p-p_n)}
$$
avec $deg(N(p)) = deg(D(p))$, $p_1\neq p_2\neq p_n$. On peut décomposer la fonction ainsi :
$$
F(p) = K + \dfrac{\alpha_1}{p-p_1} + \dfrac{\alpha_2}{p-p_2} 
+ \cdot \cdot \cdot + 
\dfrac{\alpha_n}{p-p_n} 
$$
avec $\alpha_i\in \mathbb{R}$ et $K \in \mathbb{R}$
\end{resultat}

\begin{methode}
\begin{enumerate}
\item Calcul des $\alpha_i$ :
\begin{enumerate}
\item On multiplie les deux formes de $F(p)$ par $(p-p_i)$.
\item On pose $p=p_i$.
\item On détermine $\alpha_i$.
\item On réalise l'opération $n$ fois (avec $n=deg(D(p))$.
\end{enumerate}
\item Calcul de K : 
\begin{enumerate}
\item Pour les deux formes de $F(p)$ on calcule $\lim\limits_{p\to\infty} F(p)$.
\item On détermine $K$.
\end{enumerate}
\end{enumerate}
\end{methode}

\begin{exemple}
\textit{Décomposer la fonction suivante en éléments simples :}
$$
F(p)=\dfrac{p^2+1}{(p+1)(p+4)}
$$
\end{exemple}


\section{Cas d'une fonction avec des pôles réels multiples -- $deg(N(p)) < deg(D(p))$}

\begin{resultat}
Soit la fonction rationnelle suivante :
$$
F(p) = \dfrac{N(p)}{D(p)} =\dfrac{N(p)}{(p-p_1)^n}
$$
avec $deg(N(p)) < deg(D(p))$. On peut décomposer la fonction ainsi :
$$
F(p) = \dfrac{\alpha_1}{p-p_1} + \dfrac{\alpha_2}{\left(p-p_1\right)^2} 
+ \cdot \cdot \cdot + 
\dfrac{\alpha_n}{\left(p-p_1\right)^n} 
$$
avec $\alpha_i\in \mathbb{R}$.
\end{resultat}


\begin{methode}
Les méthodes précédentes restent utilisables. Elles ne permettent pas de déterminer touts les $\alpha_i$. 

On peut alors mettre les deux formes de $F(p)$ au même dénominateur et identifier les monômes. 

On peut aussi prendre des valeurs particulières de $p$ et résoudre un système d'équations.

Le calcul de certaines limites en $\infty$ peut permettre de déterminer certains coefficients. 
\end{methode}
\begin{exemple}
\textit{Décomposer la fonction suivante en éléments simples :}
$$
F(p)=\dfrac{p^2+1}{p^3(p+1)}
$$
\end{exemple}



\section{Cas d'une fonction avec des pôles complexes -- $deg(N(p)) < deg(D(p))$}

\begin{resultat}
Soit la fonction rationnelle suivante :
$$
F(p) = \dfrac{N(p)}{D(p)} =\dfrac{N(p)}{(p-p_1)(ap^2+bp+c)}
$$
avec $deg(N(p)) < deg(D(p))$. On peut décomposer la fonction ainsi :
$$
F(p) = \dfrac{\alpha_1}{p-p_1} + \dfrac{\alpha_2 + \alpha_3 p}{ap^2+bp+c} 
$$
avec $\alpha_i\in \mathbb{R}$.
\end{resultat}



\begin{exemple}
\textit{Décomposer la fonction suivante en éléments simples :}
$$
F(p)=\dfrac{p+3}{(p+1)(p^2+1)}
$$
\end{exemple}



\end{document}
