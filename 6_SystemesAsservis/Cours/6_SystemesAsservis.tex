\documentclass[10pt,oneside]{article}
\input{style/coursHeadings}

\usepackage{style/schemabloc}
%Si le boolen xp est vrai : compilation pour xabi
%Sinon compilation Damien
\newboolean{xp}
\setboolean{xp}{true}

\newboolean{prof}
\setboolean{prof}{true}

\def\xxtitre{\ifthenelse{\boolean{xp}}{
CI 2 -- SLCI : Étude du comportement des Systèmes Linéaires Continus Invariants}{
}}

\def\xxsoustitre{\ifthenelse{\boolean{xp}}{
Chapitre 6 -- Étude des performances des systèmes complexes -- Précision -- Stabilité}{
}}


\def\xxauteur{\ifthenelse{\boolean{xp}}{
\noindent 2013 -- 2014 \\
Xavier \textsc{Pessoles}}{
}}


\def\xxpied{\ifthenelse{\boolean{xp}}{
CI 2 : SLCI -- Cours \\
Ch 6 : Performance des systèmes -- \ifthenelse{\boolean{prof}}{P}{E}%
}{
}}

\usepackage[%
    pdftitle={SLCI - Performance des systèmes},
    pdfauthor={Xavier Pessoles},
    colorlinks=true,
    linkcolor=blue,
    citecolor=magenta]{hyperref}



\usepackage{pifont}
\sloppy
\hyphenpenalty 10000


\begin{document}


\input{style/entete1}

\begin{center}
 \huge\textsc{\xxtitre}
\end{center}

\begin{center}
 \LARGE\textsc{\xxsoustitre}
\end{center}

\vspace{.5cm}



\vspace{.5cm}

\begin{center}
\begin{tabular}{ccc}
\includegraphics[width=4cm]{png/dmg} &&
%\includegraphics[height=3.5cm]{png/schema} && 
\includegraphics[width=10cm]{png/schemabloc}\\
\textit{DMU 60 eVo Linear} &&
%\textit{Schématisation du mécanisme} &&
\textit{Modélisation par schéma bloc}\\
\textit{Centre d'usinage 5 axes continus\cite{cite1}} && 
\textit{d'un axe numérique asservi \cite{cite2}}\\
\end{tabular}
\end{center}

\vspace{.2cm}

\begin{obj}
\textsc{Problématique :}
\begin{itemize}
\item La modélisation de systèmes multiphysiques donne lieu à des schémas bloc de plus en plus complexes. Comment améliorer la performance de ces systèmes en vue d'améliorer la rapidité, la stabilité ou la précision ?
%\item Comment modéliser un système complexe multiphysique en utilisant la modélisation en schéma bloc et la modélisation dans le domaine de Laplace ?
%\item Comment déterminer la fonction de transfert d'un système dans le but de prévoir son comportement ?
\end{itemize}
\end{obj}

%\begin{savoir}
%\textbf{Savoirs :}
%\begin{itemize}
%\item Mod%-C2.3 : Modèles canoniques du second ordre
%\begin{itemize}
%\item Mod%-C2-S1 : Identifier le comportement d’un système pour l’assimiler à un modèle canonique, à partir d’une réponse temporelle 
%\item Mod%-C2-S2 : Établir un modèle de comportement à partir de relevés expérimentaux
%\item Mod-C2-S3 : On pourra étudier les systèmes du premier ordre présentant un retard pur
%\end{itemize}
%\end{itemize}
%\end{savoir}

\setlength{\parskip}{0ex plus 0.2ex minus 0ex}
 \renewcommand{\contentsname}{}
 \renewcommand{\baselinestretch}{1}

% \vspace{1cm}
\textit{Ce document est en évolution permanente. Merci de signaler toutes
erreurs ou coquilles.}

\tableofcontents

 \renewcommand{\baselinestretch}{1.2}
\setlength{\parskip}{2ex plus 0.5ex minus 0.2ex}

\section{Introduction}


On s'intéresse à un système asservi classique. Les contraintes à respecter vis-à vis du cahier des charges sont des contraintes de stabilité, rapidité et de précision.

\begin{center}
\includegraphics[width=.55\textwidth]{png/bloc1}
\end{center}

\begin{exemple}
Pour modéliser l'axe asservi d'une machine outil la modélisation suivante :
\begin{center}
\includegraphics[width=.6\textwidth]{png/bloc2}
\end{center}
On prendra $K(p)=1$, $H(p)$ une fonction de transfert du premier ordre de gain $K_M$ et de constante de temps $\tau$, $G(p)=K_T$ permet de transformer la vitesse de rotation en une vitesse de translation. $C(p)$ est un correcteur de la forme $C(p)=K_C$.
\end{exemple}


\section{Étude du modèle sans perturbation}


\begin{minipage}[c]{.48\linewidth}
\begin{center}
\includegraphics[width=.95\textwidth]{png/bloc11}
\end{center}
\end{minipage}\hfill
\begin{minipage}[c]{.48\linewidth}
Dans ces conditions, exprimons la fonction de transfert en boucle fermée, la fonction de transfert en boucle ouverte et la précision du système :

$$
FTBF(p)
=\dfrac{S(p)}{E(p)}
=\dfrac{C(p)H(p)}{1+C(p)H(p)F(p)}
$$

\end{minipage}


$$
FTBO(p) = C(p) \cdot H(p) \cdot F(p)
$$


Dans tous les cas, $FTBO(p)$ est une fraction rationnelle et peut s'écrire sous la forme suivante : 

$$
FTBO(p)=\dfrac{N(p)}{D(p)}=\dfrac{K\left(1+a_1p +a_2p^2 + ... + a_m p^m \right)}{p^\alpha \left(1+b_1p +b_2p^2 + ... + b_n p^n \right)}
$$


$$
\varepsilon(p)
=\dfrac{1}{1+FTBO(p)} \cdot E(p)
=\dfrac{1}{1+C(p) \cdot H(p) \cdot F(p)} 
$$

\begin{rem}
La précision du système dépend des caractéristiques de la FTBO. On note $K$ son gain et $\alpha$ sa classe.
\end{rem}


Exprimons l'erreur du système :

$$
\lim\limits_{t\to \infty} \varepsilon(t) = \lim\limits_{p\to 0} p \varepsilon(p)
= \lim\limits_{p\to 0} p \dfrac{1}{1+FTBO(p)} \cdot E(p)
= \lim\limits_{p\to 0} p \dfrac{1}{1+\dfrac{K}{p^\alpha}} \cdot E(p)
= \lim\limits_{p\to 0} p \dfrac{p^\alpha}{p^\alpha+K} \cdot E(p)
$$


\begin{exemple}
On considère la perturbation nulle.
\ifthenelse{\boolean{prof}}{%
La fonction de transfert du moteur est de la forme $H(p)=\dfrac{K_M}{1+\tau p}$. 

On a donc :
$$
FTBF(p) = \dfrac{K_C\cdot \dfrac{K_M}{1+\tau p} \cdot K_T}{1+K_C\cdot \dfrac{K_M}{1+\tau p} \cdot K_T}
= \dfrac{K_C K_M K_T}{\left(1+\tau p\right)+K_C K_M K_T}
$$

$$
FTBO(p) = K_C\cdot \dfrac{K_M}{1+\tau p} \cdot K_T
$$


$$
\varepsilon(p)
= \dfrac{1}{1+\dfrac{K_CK_MK_T}{1+\tau p}}\cdot E(p)
= \dfrac{1+\tau p}{1+\tau p+K_CK_MK_T}\cdot E(p)
$$
$$
\varepsilon(p) 
= \dfrac{\dfrac{1}{1+K_CK_MK_T}\left( 1+\tau p\right)}{1+\dfrac{\tau}{1+K_CK_MK_T} p}\cdot E(p)
$$

}{
\rotatebox{90}{
\begin{tabular}{p{5cm}}
\\
\end{tabular}}
}
\end{exemple}

\subsection{Entrée échelon}
Dans ce cas : 
$$
\varepsilon_S =
\lim\limits_{t\to \infty} \varepsilon(t) 
= \lim\limits_{p\to 0} p \dfrac{p^\alpha}{p^\alpha+K} \cdot \dfrac{E_0}{p}
= \lim\limits_{p\to 0}  \dfrac{E_0 p^\alpha}{p^\alpha+K} 
$$

Ainsi :
\begin{itemize}
\item si $\alpha=0$ : $\varepsilon_S = \dfrac{E_0}{1+K}$. Le système est donc plus précis lorsque le gain $K$ de la FTBO augmente;
\item si $\alpha>0$ : $\varepsilon_S = 0$. L'écart statique est donc nul quel que soit $K$.
\end{itemize}

\begin{exemple}
\ifthenelse{\boolean{prof}}{%
$$
\varepsilon_S 
=\lim\limits_{t\to \infty} \varepsilon(t) 
= \lim\limits_{p\to 0} p \varepsilon(p)
= \lim\limits_{p\to 0} p \dfrac{\dfrac{1}{1+K_CK_MK_T}\left( 1+\tau p\right)}{1+\dfrac{\tau}{1+K_CK_MK_T} p}\cdot \dfrac{1}{p}
=\dfrac{1}{1+K_CK_MK_T}
$$

Pour augmenter la précision statique, il faut augmenter le gain du correcteur proportionnel $K_C$.
}{
\rotatebox{90}{
\begin{tabular}{p{4cm}}
\\
\end{tabular}}
}


\end{exemple}


\subsection{Entrée rampe}
Dans ce cas : 
$$
\varepsilon_V =
\lim\limits_{t\to \infty} \varepsilon(t) 
= \lim\limits_{p\to 0} p \dfrac{p^\alpha}{p^\alpha+K} \cdot \dfrac{E_0}{p^2}
= \lim\limits_{p\to 0}  \dfrac{p^\alpha}{p^\alpha+K} \cdot \dfrac{E_0}{p}
$$

Ainsi :
\begin{itemize}
\item si $\alpha=0$ : $\varepsilon_S = \infty$. Le système est donc instable;
\item si $\alpha=1$ : $\varepsilon_S = \dfrac{E_0}{K}$. L'écart de trainage diminue lorsque $K$ augmente;
\item si $\alpha>1$ : $\varepsilon_S = 0$. L'écart de trainage est nul.
\end{itemize}

\begin{exemple}
\ifthenelse{\boolean{prof}}{%
$$
\varepsilon_V
=\lim\limits_{t\to \infty} \varepsilon(t) 
= \lim\limits_{p\to 0} p \varepsilon(p)
= \lim\limits_{p\to 0} p \dfrac{\dfrac{1}{1+K_CK_MK_T}\left( 1+\tau p\right)}{1+\dfrac{\tau}{1+K_CK_MK_T} p}\cdot \dfrac{1}{p^2}
=+\infty
$$

Le système est donc instable lorsqu'il est soumis à une rampe. Utiliser un correcteur intégral de la forme $C(p)=\dfrac{K_C}{p}$ permettrait de stabiliser le système. En augmentant alors $K_C$, on réduirait l'erreur de traînage.

}{
\rotatebox{90}{
\begin{tabular}{p{4cm}}
\\
\end{tabular}}
}


\end{exemple}



\subsection{Entrée en accélération}
Dans ce cas : 
$$
\varepsilon_A =
\lim\limits_{t\to \infty} \varepsilon(t) 
= \lim\limits_{p\to 0} p \dfrac{p^\alpha}{p^\alpha+K} \cdot \dfrac{E_0}{p^3}
= \lim\limits_{p\to 0}  \dfrac{p^\alpha}{p^\alpha+K} \cdot \dfrac{E_0}{p^2}
$$

Ainsi :
\begin{itemize}
\item si $\alpha=0$ : $\varepsilon_A = \infty$. Le système est donc instable;
\item si $\alpha=1$ : $\varepsilon_A = \infty$. Le système est donc instable;
\item si $\alpha=2$ : $\varepsilon_A = \dfrac{E_0}{K}$. L'écart diminue lorsque $K$ augmente.
\end{itemize}

\newpage 

\subsection{Bilan}

\begin{resultat}
La précision d'un système dépend du gain $K$ et de la classe $\alpha$ de la FTBO. 
\begin{center}
\begin{tabular}{|c|c|c|c|c|}
\hline
$e(t)$ & $E(p)$ & $\alpha=0$ & $\alpha=1 $ & $\alpha=2$ \\
\hline
Échelon & $\dfrac{1}{p}$ & $\dfrac{1}{1+K}$ & 0 & 0 \\
\hline
Rampe & $\dfrac{1}{p^2}$ & $\infty$ & $\dfrac{1}{K}$ & 0 \\
\hline
Accélération & $\dfrac{1}{p^3}$ & $\infty$ & $\infty$ & $\dfrac{1}{K}$ \\
\hline
\end{tabular}
\end{center}
\end{resultat}

\begin{rem}
On montre en deuxième année que l'augmentation de $K$ ou de la classe peut être cause d'instabilité.
\end{rem}


\section{Étude du système soumis a une perturbation}
\begin{center}
\includegraphics[width=.5\textwidth]{png/bloc1.png}
\end{center}

Exprimons l'erreur du système soumis à perturbation en fonction des deux entrées:
\begin{eqnarray*}
\varepsilon(p) 
&=& E(p)-S(p)\cdot F(p) \\
&=& E(p)-\left(P(p)+\varepsilon(p)C(p)\right)\cdot F(p)H(p) \\
&=& E(p)-P(p)F(p)H(p)-\varepsilon(p)C(p)F(p)H(p) \\
\end{eqnarray*}

\begin{eqnarray*}
\Longleftrightarrow&
\varepsilon(p) \left(1+C(p)F(p)H(p) \right)=E(p)-P(p)F(p)H(p) \\
\Longleftrightarrow&
\varepsilon(p) =\dfrac{E(p)-P(p)F(p)H(p)}{1+C(p)F(p)H(p)}\\
\Longleftrightarrow&
\varepsilon(p) =
\dfrac{1}{1+C(p)F(p)H(p)}E(p)
-\dfrac{F(p)\text{Attention H(p)}}{1+C(p)F(p)H(p)}P(p)
\end{eqnarray*}

\begin{exemple}
Cas 1 : $C(p)=K_C$

\ifthenelse{\boolean{prof}}{%
$$
\varepsilon(p)=\dfrac{1}{1+G(p)H(p)C(p)}V_c(p) - \dfrac{G(p)}{1+G(p)H(p)C(p)}\Omega_r(p)
$$

$$
\varepsilon(p)=\dfrac{1+\tau p}{1+K_TK_MK_C + \tau p}V_c(p) - K_T
\dfrac{1+\tau_p}{1+K_TK_MK_C + \tau p}\Omega_r(p)
$$

Le système est soumis à une consigne échelon d'amplitude $E_c$ et à une perturbation échelon d'amplitude $E_p$. Dans ce cas, 

$$
\varepsilon(p)=\dfrac{1+\tau p}{1+K_TK_MK_C + \tau p} \dfrac{E_c}{p} - K_T
\dfrac{1+\tau_p}{1+K_TK_MK_C + \tau p} \dfrac{E_p}{p}
$$

$$
\varepsilon_S = \lim\limits_{p\to0} p \varepsilon(p)
=\lim\limits_{p\to0}  \dfrac{1+\tau p}{1+K_TK_MK_C + \tau p} E_c - K_T
\dfrac{1+\tau_p}{1+K_TK_MK_C + \tau p} E_p
$$
$$
\varepsilon_S 
= \dfrac{E_c -  K_T E_p}{1+K_TK_MK_C} 
$$

L'augmentation du gain du correcteur permet de diminuer l'écart statique.

}{
\rotatebox{90}{
\begin{tabular}{p{8cm}}
\\
\end{tabular}}
}
\end{exemple}

\begin{exemple}
Cas 2 : $C(p)=\dfrac{K_C}{p}$


\ifthenelse{\boolean{prof}}{%

}{
\rotatebox{90}{
\begin{tabular}{p{8cm}}
\\
\end{tabular}}
}
\end{exemple}

\section{Étude de la stabilité des systèmes}
Il ne s'agit pas ici de faire une étude exhaustive de la stabilité des systèmes asservis, mais d'avoir une idée sur un des critères de stabilité
\subsection{Système d'ordre 1}
Soit un système du premier ordre sous sa forme canonique :
$$
H(p)=\dfrac{K}{1+\tau p}
$$

Sa réponse temporelle à une entrée échelon d'amplitude $E_0$ est donnée par :
$$
s(t) = E_0 K \left(1-e^{-\dfrac{t}{\tau}}\right)
$$

Le système est instable si $\tau$ est négatif.

\subsection{Système d'ordre 2}
Soit un système du second ordre :
$$
H(p)
=\dfrac{K}{1+\dfrac{2\xi}{\omega_0} p+\dfrac{1}{\omega_0^2} p^2}
=\dfrac{K\omega_0^2}{\omega_0^2+2\xi\omega_0 p+ p^2}
$$


On le sollicite par un échelon d'amplitude $E_0$. En conséquence, $E(p)=\dfrac{E_0}{p}$ et on a :
$$
S(p)=\dfrac{E_0}{p} \cdot \dfrac{K\omega_0^2}{\omega_0^2+2\xi\omega_0 p+ p^2}
$$

\subsubsection{Cas où $\xi>1$}
Dans ce cas, $\omega_0^2+2\xi\omega_0 p+ p^2$ peut se factoriser sous la forme $\left(p-p_1\right)\cdot\left(p-p_2\right)$ avec $p_1 = -\xi\omega_0 + \omega_0 \sqrt{\xi^2-1}$ et $p_2 = -\xi\omega_0 - \omega_0 \sqrt{\xi^2-1}$.

On montre donc que :
$$
S(p)= \dfrac{KE_0}{p} + \dfrac{KE_0 p_2}{p_1-p_2}\cdot \dfrac{1}{p-p_1}
+ \dfrac{KE_0 p_1}{p_2-p_1}\cdot \dfrac{1}{p-p_2}
$$

Et alors, 
$$
s(t) = KE_0\left(1+ \dfrac{ p_2}{p_1-p_2} e^{p_1 \cdot t} + \dfrac{p_1}{p_2-p_1}\cdot e^{p_2 t} \right) \cdot u(t)
$$


Dans ce cas, 

\begin{center}
 \includegraphics[width=.8\textwidth]{png/poles_1}
\end{center}

\subsubsection{Cas où $\xi=1$}
Dans ce cas, $\omega_0^2+2\xi\omega_0 p+ p^2$ se met sous la forme $\left( p-p_1^2\right) $ avec $p_1 = -\omega_0$.

On montre donc que :
$$
S(p)= \dfrac{KE_0}{p} - \dfrac{KE_0}{p-p_1} - \dfrac{KE_0 \omega_0}{\left(p-p_1\right)^2}
$$

Et alors, 
$$
s(t) = KE_0\left(1- e^{-\omega_0 \cdot t} -\omega_0 t  e^{-\omega_0t} \right) \cdot u(t)
$$

\subsubsection{Cas où $\xi<1$}
Dans ce cas, $\omega_0^2+2\xi\omega_0 p+p^2$ peut se factoriser sous la forme $\left(p-p_1\right)\cdot\left(p-p_2\right)$ avec $p_1 = -\xi\omega_0 + j \omega_0 \sqrt{1-\xi^2}$ et $p_2 = -\xi\omega_0 - j \omega_0 \sqrt{1-\xi^2}$.

On montre donc que :
$$
S(p)= KE_0\left( 
\dfrac{1}{p}
-\dfrac{p+\xi\omega_0}{\left( p+\xi\omega_0\right)^2+\omega_0^2\left(1-\xi^2\right)}
-\dfrac{\xi\omega_0}{\left( p+\xi\omega_0\right)^2+\omega_0^2\left(1-\xi^2\right)}
\right)
$$

Et alors, 
$$
s(t) = KE_0\left(1
- e^{-\xi\omega_0 t}\cdot\cos \left(t\omega_0 \sqrt{1-\xi^2} \right)
- \dfrac{\xi}{\sqrt{1-\xi^2}} e^{-\xi\omega_0 t}\cdot\sin \left(t\omega_0 \sqrt{1-\xi^2} \right)
\right) \cdot u(t)
$$


\begin{center}
 \includegraphics[width=.8\textwidth]{png/poles_3}
\end{center}

%\begin{exemple}
%Cas 3 : $C(p)=K_C$, $C(p)=\dfrac{K_T}{p}$
%\ifthenelse{\boolean{prof}}{
%\rotatebox{90}{
%\begin{tabular}{p{8cm}}
%\\
%\end{tabular}}
%}
%\end{exemple}

\begin{resultat}
On montre qu'un système est stable si les pôles de la \textbf{FTBF} sont à partie réelle \textbf{strictement négative}.
\end{resultat}

\begin{thebibliography}{2}
   \bibitem[1]{cite1} DMU 60 eVo linear, \textit{DMG -- Deckel Maho -- Gildemeiseter}, \url{http://fr.dmg.com}.
   \bibitem[2]{cite2} Programmation des machines-outils à commande numérique (MOCN), \textit{Étienne Lefur et Christophe Sohier}, École Normale Supérieure de Cachan, \url{http://etienne.lefur.free.fr/}.
   \bibitem[3]{cite3} SLCI : Systèmes asservis en boucle fermée : stabilité et précision, \textit{Joël Boiron}, PTSI -- Lycée Gustave Eiffel de Bordeaux.

\end{thebibliography}

\end{document}
