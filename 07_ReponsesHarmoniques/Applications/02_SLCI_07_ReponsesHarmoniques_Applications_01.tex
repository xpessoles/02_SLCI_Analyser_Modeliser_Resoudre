\documentclass[10pt]{article}
\input{style/coursHeadings}
\input{style/programHeadings}
\input{style/macros_SII}
\input{style/macros_Titres}
\input{style/macros_Frames}

%Si le boolen xp est vrai : compilation pour xabi
%Sinon compilation Damien

\newif\ifprof
%\proftrue
\proffalse

\newif\ifxp
\xptrue
%\xpfalse

\newif\iftd
\tdtrue
%\tdfalse


\usepackage[%
    pdftitle={},
    pdfauthor={Xavier Pessoles},
    colorlinks=true,
    linkcolor=blue,
    citecolor=magenta]{hyperref}


\def\discipline{Sciences Industrielles de l'Ingénieur}
\def\xxtitre{%
\ifxp
Partie 2 : SLCI -- Analyser le comportement des Systèmes Linéaires Continus Invariants
\else
\fi
}

\def\xxsoustitre{%
\ifxp
Chapitre 7 -- Réponses harmoniques -- Diagrammes de Bode
\else
\fi}

\def\xxauteur{%
\ifxp
Xavier \textsc{Pessoles}
\else
\fi}

\def\xxpied{%
\ifxp
Partie 2 : SLCI\\
Ch. 7 : Réponses harmoniques -- Applications 01
\else
\fi}



%---------------------------------------------------------------------------


\begin{document}
\ifxp
\input{style/enteteXP}
\else
\input{style/enteteDI}
\fi




 \renewcommand{\baselinestretch}{1.2}
\setlength{\parskip}{2ex plus 0.5ex minus 0.2ex}



\subsection*{Exercice 1 : Identification du comportement des systèmes}

On donne le diagramme de Bode d'une fonction de transfert :

\begin{center}
\includegraphics[width=.7\linewidth]{images/identification_1.png}
\end{center}

\subparagraph{}\textit{Expérimentalement, avec quel type de signal doit-on solliciter un système pour tracer un diagramme de Bode ?}
\subparagraph{} \textit{Pour des pulsations de 1 rad/s, 15 rad/s et 100 rad/s, tracer l'allure du signal d'entrée et le signal de sortie. Calculer (si nécessaire) la pulsation, l'amplitude et le déphasage du signal de sortie. }

\subparagraph{}\textit{Tracer le diagramme asymptotique. De quel type de système s'agit-il ? Donner sa forme canonique. Justifier.}
\subparagraph{} \textit{Identifier les différentes constantes de ce système.}

\ifprof
\begin{corrige}
On a : $K=12$, $\xi=0,15$ et $\omega_0 = 15 \; \text{s}^{-1}$.
\end{corrige}
\else
\fi




\subsection*{Exercice 2 : Tracé de diagramme de Bode}
\subparagraph*{}\textit{Tracer le diagramme de Bode des fonctions suivantes :}
$$
H(p)=\dfrac{10}{1+0,1p} \quad \quad 
H(p)=\dfrac{100}{p(1+p)} \quad \quad 
H(p)=\dfrac{10}{1+0,01p + p^2} \quad \quad
H(p)=\dfrac{1}{p(1+p)(1+0,1p)(1+0,01p)} 
$$

\begin{center}
\includegraphics[width=.9\textwidth]{images/bode_vierge.pdf}
\end{center}


\begin{center}
\includegraphics[width=.9\textwidth]{images/bode_vierge.pdf}
\end{center}


\begin{center}
\includegraphics[width=.9\textwidth]{images/bode_vierge.pdf}
\end{center}



\begin{center}
\includegraphics[width=.9\textwidth]{images/bode_vierge.pdf}
\end{center}




\end{document}


