\documentclass[10pt,oneside]{article}
\input{style/coursHeadings}

\usepackage{style/schemabloc}
%Si le boolen xp est vrai : compilation pour xabi
%Sinon compilation Damien
\newboolean{xp}
\setboolean{xp}{true}

\newboolean{prof}
\setboolean{prof}{true}

\def\xxtitre{\ifthenelse{\boolean{xp}}{
CI 2 -- SLCI : Étude du comportement des Systèmes Linéaires Continus Invariants}{
}}

\def\xxsoustitre{\ifthenelse{\boolean{xp}}{
Chapitre 7 -- Réponses harmoniques -- Diagrammes de Bode}{
}}


\def\xxauteur{\ifthenelse{\boolean{xp}}{
\noindent 2013 -- 2014 \\
Xavier \textsc{Pessoles}}{
}}


\def\xxpied{\ifthenelse{\boolean{xp}}{
CI 2 : SLCI -- Cours \\
Ch 7 : Réponses harmoniques -- \ifthenelse{\boolean{prof}}{P}{E}%
}{
}}

\usepackage[%
    pdftitle={SLCI - Réponses harmoniques},
    pdfauthor={Xavier Pessoles},
    colorlinks=true,
    linkcolor=blue,
    citecolor=magenta]{hyperref}



\usepackage{pifont}
\sloppy
\hyphenpenalty 10000


\begin{document}


\input{style/entete1}

\begin{center}
 \Large\textsc{\xxtitre}
\end{center}

\begin{center}
 \large\textsc{\xxsoustitre}
\end{center}

\begin{center}
 \large\textsc{Exercices d'application}
\end{center}
\vspace{.5cm}


\subsection*{Exercice 1 : Tracé de diagramme de Bode}
\textit{Tracer le diagramme de Bode des fonctions suivantes :}
\begin{eqnarray*}
H(p)&=&\dfrac{3}{0,02p^2 + 0,2 p + 2} \\
H(p)&=&\dfrac{7}{p(1+p)} \\
H(p)&=&\dfrac{67,5 \left(p+0,4\right)}{p\left(p+3 \right)\left(p+10 \right)} \\
H(p)&=&\dfrac{0,05 p +2 }{0,1 p +1} \\
H(p)&=&\dfrac{1+2p }{9+2p+p^2} 
\end{eqnarray*}

\begin{center}
\rotatebox{90}{\includegraphics[width=1.1\textheight]{png/bode1.png}}
\end{center}

\begin{center}
\rotatebox{90}{\includegraphics[width=1.1\textheight]{png/bode2.png}}
\end{center}

\begin{center}
\rotatebox{90}{\includegraphics[width=1.1\textheight]{png/bode3.png}}
\end{center}

\begin{center}
\rotatebox{90}{\includegraphics[width=1.1\textheight]{png/bode4.png}}
\end{center}

\begin{center}
\rotatebox{90}{\includegraphics[width=1.1\textheight]{png/bode5.png}}
\end{center}




\end{document}
 