\documentclass[10pt]{article}
\input{style/coursHeadings}
\input{style/programHeadings}
\input{style/macros_SII}
\input{style/macros_Titres}
\input{style/macros_Frames}

%Si le boolen xp est vrai : compilation pour xabi
%Sinon compilation Damien
\newboolean{xp}
\setboolean{xp}{true}

\newboolean{prof}
\setboolean{prof}{true}

\usepackage[%
    pdftitle={Représentation des nombres},
    pdfauthor={Xavier Pessoles},
    colorlinks=true,
    linkcolor=blue,
    citecolor=magenta]{hyperref}


\def\discipline{Sciences Industrielles de l'Ingénieur}
\def\xxtitre{\ifthenelse{\boolean{xp}}{
CI 2 -- SLCI : Étude du comportement des Systèmes Linéaires Continus Invariants}{
Chapitre  -- }}

\def\xxsoustitre{\ifthenelse{\boolean{xp}}{
Chapitre 7 -- Réponses harmoniques -- Diagrammes de Bode}{
Partie  -- }}

\def\xxauteur{\ifthenelse{\boolean{xp}}{
Xavier \textsc{Pessoles} \\
2013 -- 2014}{
Damien \textsc{Iceta} \\ Xavier \textsc{Pessoles}}}

\def\xxpied{\ifthenelse{\boolean{xp}}{
CI 2 -- Étude des SLCI\\
Ch. 7 : Réponses Harmoniques -- TD}{
\xxtitre}}

\def\xxcathegorie{\ifthenelse{\boolean{xp}}{
2013 -- 2014 \\
Xavier \textsc{Pessoles}}{
}}





%---------------------------------------------------------------------------


\begin{document}

\ifthenelse{\boolean{xp}}{\input{style/enteteXP}}{\input{style/enteteDI}}
\begin{center}
\large{\textsc{Travaux Dirigés}}
\end{center}

\begin{flushright}
\textit{Ressources de Florestan Mathurin.}
\end{flushright}

 \renewcommand{\baselinestretch}{1.2}
\setlength{\parskip}{2ex plus 0.5ex minus 0.2ex}

\subsection*{Exercice 1 : Radar d'avion}

\begin{minipage}[c]{.78\linewidth}
Le support d'étude est un radar d'avion. Il permet au pilote de connaître la position des engins extérieurs (avions, hélicoptères, bateaux...). L'objectif de cette étude est de vérifier les exigences décrites dans le cahier des charges. 
\end{minipage} \hfill
\begin{minipage}[c]{.18\linewidth}
\begin{center}
\includegraphics[width=.95\textwidth]{images/exo1_1}
\end{center}
\end{minipage}


\begin{center}
\includegraphics[width=.8\textwidth]{images/ExigencesRadar}
\end{center}

On réalise un asservissement de la position angulaire du radar d'avion : l'angle souhaité est $\theta_c(t)$ l'angle réel du radar est $\theta_r(t)$. La différences des deux angles est transformée en une tension $u_m(t)$, selon la loi $u_m(t)=A\cdot\left( \theta_c(t) -\theta_r(t) \right)$. La tension $u_m(t)$ engendre, via un moteur de fonction de transfert $H_m(t)$, une  vitesse angulaire $\omega_m(t)$. Cette vitesse angulaire est réduite grâce à un réducteur de vitesse, selon la relation $\omega_r(t)=B\omega_m(t)$ ($B<1$), $\omega_r(t)$ étant la vitesse angulaire du radar.

 On donne la relation $\omega_r(t)= \dfrac{d\theta_r(t)}{dt}$.
 
 
 \subparagraph{}
 \textit{Réaliser le schéma bloc du système.}
 
 Les équations du moteur à courant continu qui est utilisé dans la motorisation sont les suivantes :
 $$
 u_m(t)=e(t)+Ri(t) 
 \quad e(t)=k_e\omega_m(t) 
 \quad J\dfrac{d\omega_m(t)}{dt}=C_m(t)
 \quad C_m(t)=k_mi(t)
 $$
 Avec : 
 
 \begin{itemize}
 \item $u(t)$ : tension aux bornes du moteur (en V) (entrée du moteur);
 \item $e(t)$ : force contre électromotrice (en V);
 \item $\omega_m(t)$ : vitesse de rotation du moteur (en $rad/s$);
 \item $C_m(t)$ : couple moteur (en $N\cdot m$);
 \item $J$ : inertie équivalente en rotation de l'arbre moteur (en $kg\cdot m^2$);
 \item $R$ : résistance électrique du moteur;
 \item $k_e$ : constante de force contre électromotrice;
 \item $k_m$ : constante de couple.
 \end{itemize}
 
 \subparagraph{}
 \textit{Déterminer la fonction de transfert $H_m(p)=\dfrac{\Omega_m(p)}{U_m(p)}$.}
 
 \subparagraph{}
 \textit{Montrer que $H_m(p)$ peut se mettre sous la forme canonique $H_m(p)=\dfrac{K_m}{1+T_m p}$. Déterminer les valeurs littérales de $K_m$ et $T_m$.}
 
 
 \subparagraph{}
 \textit{Déterminer $\omega_m(t)$ lorsque $u_m(t)$ est un échelon de tension d'amplitude $u_0$. Exprimer le résultat en fonction de $K_m$, $T_m$ et $u_0$. Préciser la valeur de $\omega_m(t)$ à l'origine, la pente de la tangente à l'origine de $\omega_m(t)$ et la valeur finale atteinte par $\omega_m(t)$ quand $t$ tend vers l'infini.}
 
 
 \subparagraph{}
 \textit{Déterminer la fonction de transfert $H(p)=\dfrac{\theta_r(p)}{\theta_c(p)}$. Montrer que cette fonction peut se mettre sous la forme canonique d'un système du second ordre où les différentes constantes seront à déterminer.}
 
 La réponse indicielle de $H(p)$ à un échelon unitaire est donnée sur la figure suivante : 
 
 \begin{center}
\includegraphics[width=.8\textwidth]{images/exo1_2}
\end{center}
 
 \subparagraph{}
 \textit{Déterminer, en expliquant la démarche utilisée, les valeurs numériques du gain, du coefficient d'amortissement et de la pulsation propre du système.}
 
 Pour la suite, on prendre : $K=1$, $\xi=0,5$ et $\omega_0 = 15\; rad/s$.
 
 
 
 \subparagraph{}
 \textit{Déterminer en expliquant la démarche utilisée le temps de réponse à 5\%. Conclure quant à la capacité du radar à vérifier l'exigence de rapidité.}
 
 On améliore la performance du radar en ajoutant un correcteur entre l'amplificateur et le moteur. 
 La nouvelle fonction de transfert est : 
 $$
 H(p)=\dfrac{1}{\left(1+0,05p \right)\left(1+0,0005p \right)\left(1+0,002p \right)}
 $$
 \subparagraph{}
 \textit{Tracer le diagramme de Bode asymptotique de cette fonction de transfert en expliquant la démarche utilisée. }
 
 
 \subparagraph{}
 \textit{Déterminer le gain et la phase pour une pulsation de 10 $rad/s$.}

\subparagraph{}
\textit{Déterminer, en régime permanent, $\theta_r(t)$ pour une entrée $\theta_c(t)=0,2 \sin (10t)$.}

Pour $\omega<20\; rad/s$, on a $H(p)\simeq \dfrac{1}{1+0,005p}$.


\subparagraph{}
\textit{Déterminer, sur cette approximation, la pulsation de coupure à $-3dB$. Conclure quant à la capacité du radar à satisfaire l'exigence de bande passante.}


 \begin{center}
\includegraphics[width=.8\textwidth]{images/exo1_3}

\includegraphics[width=.8\textwidth]{images/exo1_4}
\end{center}


\subsection*{Exercice 2 : Étude d'une antenne parabolique}
\setcounter{subparagraph}{0}
\begin{minipage}[c]{.68\linewidth}
La réception des chaînes de télévision par satellite nécessite un récepteur / décodeur et une antenne parabolique. 

Pour augmenter le nombre de chaînes reçues, l'antenne doit pouvoir s'orienter vers plusieurs satellites différents. Le satellite choisi dépend de la chaîne demandée. Tous les satellites de radiodiffusion sont situés sur l'orbite géostationnaire à 36 000 km au dessus de l'équateur. Le réglage de l'orientation de l'antenne ne nécessite donc qu'une seule rotation autour d'un axe appelé axe d'azimut. Le cahier des charges partiel à satisfaire est fourni. L'objectif est de satisfaire l'exigence d'orientation du cahier des charges.

\end{minipage} \hfill
\begin{minipage}[c]{.3\linewidth}
\begin{center}
\includegraphics[width=\textwidth]{images/exo2_1}
\end{center}
\end{minipage}

\begin{center}
\includegraphics[width=.8\textwidth]{images/ExigencesAntenne}
\end{center}

\begin{minipage}[c]{.68\linewidth}
L'axe d'orientation d'azimut utilise un dispositif de réduction de vitesse (engrenages et roue et vis sans fin). Si on note $\omega_a(t)$ la vitesse de rotation de l'axe d'orientation et $\omega_m(t)$ la vitesse de rotation de l'axe d'orientation et $\omega_m(t)$ la vitesse de rotation du moteur, on a la relation suivante : $\dfrac{\omega_a(t)}{\omega_m(t)}=\dfrac{1}{N}=\dfrac{1}{23\; 328}$. 

\end{minipage} \hfill
\begin{minipage}[c]{.3\linewidth}
\begin{center}
\includegraphics[width=\textwidth]{images/exo2_2}
\end{center}
\end{minipage}


Les équations du moteur à courant continu qui est utilisé dans la motorisation sont les suivantes :
$$
 u_m(t)=e_m(t)+R_mi_m(t)+L_m\dfrac{di_m}{dt} 
 \quad e_m(t)=K_e\omega_m(t) 
 \quad J_m\dfrac{d\omega_m(t)}{dt}=C_m(t)
 \quad C_m(t)=K_ci_m(t)
 $$
 Avec : 
 
 \begin{itemize}
 \item $u_m(t)$ : tension aux bornes du moteur (en V);
 \item $e_m(t)$ : force contre électromotrice (en V);
 \item $i_m(t)$ : intensité (en A);
 \item $\omega_m(t)$ : vitesse de rotation du moteur (en $rad/s$);
 \item $C_m(t)$ : couple moteur (en $N\cdot m$);
 \item $J_m$ : inertie équivalente en rotation de l'arbre moteur (en $kg\cdot m^2$);
 \item $R_m$ : résistance électrique du moteur (9,1 $\Omega$);
 \item $K_e$ : constante de force contre électromotrice (0,022 $V\cdot rad^{-1}\cdot s$);
 \item $K_c$ : constante de couple (0,022 $N\cdot m\cdot A^{-1}$).
 \end{itemize}
   
\subparagraph{}
\textit{Exprimer ces équations dans le domaine de Laplace. Toutes les conditions initiales seront nulles, et considérées comme telles dans la suite de l'exercice.}
  
\subparagraph{}
\textit{Réaliser le schéma-bloc moteur.}
  
\subparagraph{}
\textit{Déterminer la fonction de transfert $H(p)=\dfrac{\Omega_m(p)}{U_m(p)}$. Montrer que $H(p)$ peut se mettre sous la forme canonique $H(p)=\dfrac{K}{1+\dfrac{2\xi}{\omega_0}p+\dfrac{p^2}{\omega_0^2}}$ et déterminer les valeurs littérales de $K$, $\xi$ et $\omega_0$ en fonction des constantes fournies.}
  
On note $\tau_e = \dfrac{L_m}{R_m}$ la constante de temps électrique du moteur et $\tau_m=\dfrac{R_m J_m}{K_cK_e}$. On suppose que le temps d'établissement du courant est bien inférieur au temps de mise en mouvement de toute la mécanique, ce qui revient à dire que $\tau_e << \tau_m$. 
  
  
\subparagraph{}
\textit{Montrer alors que la fonction de transfert du moteur peut s'écrire $H(p)\simeq\dfrac{K}{\left(1+\tau_e p\right)\left(1+\tau_m p\right)}$}
  
  
\subparagraph{}
\textit{Tracer les diagrammes asymptotiques de Bode du moteur sur le document suivant. Les courbes correspondent au diagramme de Bode obtenu expérimentalement. Préciser sur ces diagrammes l'ensemble des caractéristiques (pulsations caractéristiques, pentes caractéristiques, valeurs numériques caractéristiques) connues à ce stade.}
  
\subparagraph{}
\textit{Déterminer $J_m$ et $L_m$. Justifier a posteriori que $\tau_e << \tau_m$.}
  
On soumet le moteur à un échelon de tension d'amplitude $U_0$ : $u_m(t)=U_0 u(t)$. 
  
\subparagraph{}
\textit{Justifier que la fonction $\omega_m(t)$ aura une tangente à l'origine horizontale.}

Grâce à la propriété $\tau_e<<\tau_m$, on approxime, dans toute la suite, la fonction $H(p)$ par $\dfrac{K}{1+\tau_m p}$.

  
\subparagraph{}
\textit{Déterminer l'expression analytique de $\omega_m(t)$ en fonction de $K$, $\tau_m$ et $U_0$. }

Indépendamment des résultats précédents, on prend pour la suite $\tau_m=0,012\; s.$ et $K=45\; rad\cdot s^{-1}\cdot V^{-1}$. La tension nominale d'utilisation est $U_0=18\; V$.


\subparagraph{}
\textit{Montrer que le moteur n'excède pas sa valeur limite de rotation qui es de 8000 $tr/min$.}

La chaîne d'asservissement complète est donnée sur le schéma bloc suivant ($\theta_{ac}$ est l'angle de consigne que l'on souhaite faire prendre à l'antenne; $\theta_a$ réel de l'antenne, défini par $\omega_a(t)=\dfrac{d\theta_a(t)}{dt}$; $K_a$ est un gain constant). 

\begin{center}
\includegraphics[width=.8\textwidth]{images/exo2_3}
\end{center}

\subparagraph{}
\textit{Déterminer l'expression de $G(p)$ et $M(p)$.}

\subparagraph{}
\textit{Déterminer la fonction de transfert $\dfrac{\theta_a(p)}{\theta_{ac}(p)}$, montrer que c'est une fonction du deuxième ordre et déterminer l'expression littérale de son gain $K_T$, de son coefficient d'amortissement $\xi_T$ et de sa pulsation propre non amortie $\omega_{OT}$.}

\subparagraph{}
\textit{Montrer que le système vérifie le critère d'écart de positionnement du cahier des charges.}

\subparagraph{}
\textit{Déterminer $K_a$ pour que le système puisse satisfaire le critère de temps de réponse du cahier des charges.}

\begin{center}
\includegraphics[width=.8\textwidth]{images/exo2_4}
\end{center}
\end{document}


