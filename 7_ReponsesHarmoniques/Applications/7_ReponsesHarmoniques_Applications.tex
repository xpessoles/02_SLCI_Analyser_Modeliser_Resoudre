\documentclass[10pt,oneside]{article}
\input{style/coursHeadings}

\usepackage{style/schemabloc}
%Si le boolen xp est vrai : compilation pour xabi
%Sinon compilation Damien
\newboolean{xp}
\setboolean{xp}{true}

\newboolean{prof}
\setboolean{prof}{true}

\def\xxtitre{\ifthenelse{\boolean{xp}}{
CI 2 -- SLCI : Étude du comportement des Systèmes Linéaires Continus Invariants}{
}}

\def\xxsoustitre{\ifthenelse{\boolean{xp}}{
Chapitre 7 -- Réponses harmoniques -- Diagrammes de Bode}{
}}


\def\xxauteur{\ifthenelse{\boolean{xp}}{
\noindent 2013 -- 2014 \\
Xavier \textsc{Pessoles}}{
}}


\def\xxpied{\ifthenelse{\boolean{xp}}{
CI 2 : SLCI -- Cours \\
Ch 7 : Réponses harmoniques -- \ifthenelse{\boolean{prof}}{P}{E}%
}{
}}

\usepackage[%
    pdftitle={SLCI - Réponses harmoniques},
    pdfauthor={Xavier Pessoles},
    colorlinks=true,
    linkcolor=blue,
    citecolor=magenta]{hyperref}



\usepackage{pifont}
\sloppy
\hyphenpenalty 10000


\begin{document}


\input{style/entete1}

\begin{center}
 \Large\textsc{\xxtitre}
\end{center}

\begin{center}
 \large\textsc{\xxsoustitre}
\end{center}

\begin{center}
 \large\textsc{Exercices d'application}
\end{center}
\vspace{.5cm}


\subsection*{Exercice 1 : Tracé de diagramme de Bode}
\textit{Tracer le diagramme de Bode des fonctions suivantes :}
$$
H(p)=\dfrac{10}{1+0,1p} \quad \quad 
H(p)=\dfrac{100}{p(1+p)} \quad \quad 
H(p)=\dfrac{10}{1+0,01p + p^2} \quad \quad
H(p)=\dfrac{1}{p(1+p)(1+0,1p)(1+0,01p)} 
$$

\begin{center}
\includegraphics[width=.9\textwidth]{png/bode_vierge.pdf}
\end{center}


\begin{center}
\includegraphics[width=.9\textwidth]{png/bode_vierge.pdf}
\end{center}


\begin{center}
\includegraphics[width=.9\textwidth]{png/bode_vierge.pdf}
\end{center}



\begin{center}
\includegraphics[width=.9\textwidth]{png/bode_vierge.pdf}
\end{center}

\newpage

\subsection*{Exercice 2 : Identification du comportement des systèmes}

On donne le diagramme de Bode d'une fonction de transfert :

\begin{center}
\includegraphics[width=.85\linewidth]{png/identification_1.pdf}
\end{center}


\begin{enumerate}
\item \textit{Expérimentalement, avec quel type de signal doit-on solliciter un système pour tracer un diagramme de Bode ?}
\item \textit{Tracer le diagramme asymptotique. De quel type de système s'agit-il ? Donner sa forme canonique. Justifier.}
\item %\subparagraph{}
\textit{Identifier les différentes constantes de ce système.}
\end{enumerate}

\end{document}
 