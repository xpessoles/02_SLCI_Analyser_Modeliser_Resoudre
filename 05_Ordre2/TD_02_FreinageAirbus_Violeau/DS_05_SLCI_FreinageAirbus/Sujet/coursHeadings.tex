\usepackage[T1]{fontenc}
\usepackage[utf8]{inputenc}
% \usepackage{lmodern}
%\usepackage[adobe-utopia,uppercase=upright,greeklowercase=upright]{mathdesign}
\usepackage[adobe-utopia]{mathdesign}
% \usepackage{pifont}
% \usepackage{amssymb}
\usepackage{amsmath}
\usepackage[francais]{babel}
% \usepackage[francais]{varioref}
\usepackage[dvips]{graphicx}


\usepackage{fancyhdr}
\usepackage{titlesec}
\usepackage{vmargin}
\usepackage{longtable}

\usepackage{ifthen}


%\usepackage{epsfig}
\usepackage{subfig}

\usepackage{multirow}
%\usepackage{multicol} % Portions de texte en colonnes
\usepackage{flafter}%floatants apr�s la r�f�rence



\usepackage{color}
\usepackage{colortbl}

\definecolor{gris25}{gray}{0.75}
\definecolor{bleu}{RGB}{18,33,98}
% \usepackage{pstricks}
%\usepackage{minitoc}
% \setcounter{minitocdepth}{4}

\setcounter{tocdepth}{2}

% \mtcselectlanguage{french} 

%\usepackage{draftcopy}% "Brouillon"
% \usepackage{floatflt}
\usepackage{psfrag}
%\usepackage{listings} % Permet d'ins�rer du code de programmation
\renewcommand{\baselinestretch}{1.2}

% Changer la num�rotation des figures :
% ------------------------------------
% \makeatletter
% \renewcommand{\thefigure}{\ifnum \c@section>\z@ \thesection.\fi
%  \@arabic\c@figure}
% \@addtoreset{figure}{section}
% \makeatother
 
 % Pour les vecteurs
 \newcommand{\vect}[1]{\overrightarrow{#1}}


%  ------------------------------------------
% | Modification du formatage des sections : | 
%  ------------------------------------------

% Grands titres :
% ---------------

\newcommand{\titre}[1]{%
\begin{center}
      \bigskip
      \rule{\textwidth}{1pt}
      \par\vspace{0.1cm}
      
      \textbf{\large #1}
      \par\rule{\textwidth}{1pt}
    \end{center}
    \bigskip
  }

% Supprime le numéro du chapitre dans la numérotation des sections:
% -----------------------------------------------------------------
\makeatletter
\renewcommand{\thesection}{\@arabic\c@section}
\makeatother


% \titleformat{\chapter}[display]
% {\normalfont\Large\filcenter}
% {}
% {1pc}
% {\titlerule[1pt]
%   \vspace{1pc}%
%   \Huge}[\vspace{1ex}%
% \titlerule]


%%%% Chapitres Comme PY Pechard %%%%%%%%%
% numéro du chapitre
\DeclareFixedFont{\chapnumfont}{OT1}{phv}{b}{n}{80pt}
% pour le mot « Chapitre »
\DeclareFixedFont{\chapchapfont}{OT1}{phv}{m}{it}{40pt}
% pour le titre
\DeclareFixedFont{\chaptitfont}{T1}{phv}{b}{n}{25pt}

\definecolor{gris}{gray}{0.75}
\titleformat{\chapter}[display]%
	{\sffamily}%
	{\filleft\chapchapfont\color{gris}\chaptertitlename\
	\\
	\vspace{12pt}
	\chapnumfont\thechapter}%
	{16pt}%
	{\filleft\chaptitfont}%
	[\vspace{6pt}\titlerule\titlerule\titlerule]

%%%%  Fin Chapitres Comme PY Pechard %%%%%%%%%


% Section, subsection, subsubsection sans serifs :
% % ----------------------------------------------

% \makeatletter
% \renewcommand{\section}{\@startsection{section}{0}{0mm}%
% {\baselineskip}{.3\baselineskip}%
% {\normalfont\sffamily\Large\textbf}}%
% \makeatother

\makeatletter
\renewcommand{\@seccntformat}[1]{{\textcolor{bleu}{\csname
the#1\endcsname}\hspace{0.5em}}}
\makeatother

\makeatletter
\renewcommand{\section}{\@startsection{section}{1}{\z@}%
                       {-4ex \@plus -1ex \@minus -.4ex}%
                       {1ex \@plus.2ex }%
                       {\normalfont\Large\sffamily\bfseries}}%
\makeatother
 
\makeatletter
\renewcommand{\subsection}{\@startsection {subsection}{2}{\z@}
                          {-3ex \@plus -0.1ex \@minus -.4ex}%
                          {0.5ex \@plus.2ex }%
                          {\normalfont\large\sffamily\bfseries}}
\makeatother
 
\makeatletter
\renewcommand{\subsubsection}{\@startsection {subsubsection}{3}{\z@}
                          {-2ex \@plus -0.1ex \@minus -.2ex}%
                          {0.2ex \@plus.2ex }%
                          {\normalfont\large\sffamily\bfseries}}
\makeatother
 
\makeatletter             
\renewcommand{\paragraph}{\@startsection{paragraph}{4}{\z@}%
                                    {-2ex \@plus-.2ex \@minus .2ex}%
                                    {0.1ex}%               
{\normalfont\sffamily\bfseries Question }}
\makeatother

\renewcommand{\theparagraph}{\arabic{paragraph}} 

\makeatletter
\renewcommand{\subparagraph}{\@startsection{subparagraph}{5}{\z@}%
                                       {-2ex \@plus-.1ex \@minus .2ex}%
                                       {0.1ex}%
				    {\normalfont\normalsize\sffamily\bfseries}}
\makeatletter
% \makeatletter
% \renewcommand{\subsection}{\@startsection{subsection}{1}{2mm}%
% {\baselineskip}{.3\baselineskip}%
% {\normalfont\sffamily\large\textbf}}%
% \makeatother
% 
% \makeatletter
% \renewcommand{\subsubsection}{\@startsection{subsubsection}{2}{4mm}%
% {\baselineskip}{.15\baselineskip}%
% {\normalfont\sffamily\large\textbf}}%
% \makeatother
% 
% \makeatletter
% \renewcommand{\paragraph}{\@startsection{paragraph}{3}{6mm}%
% {\baselineskip}{.15\baselineskip}%
% {\normalfont\sffamily\large\textbf}}%
% \makeatother
 
\setcounter{secnumdepth}{4}


%  --------
% | Marges |
%  --------


% \setmarginsrb{2.5cm}{1.5cm}{2.5cm}{2cm}{1cm}{1cm}{1cm}{1cm}
\setmarginsrb{1.5cm}{1cm}{1.5cm}{1cm}{1cm}{1cm}{1cm}{1cm}

% Changer les marges localement :
% -----------------------------
\newenvironment{changemargin}[2]{\begin{list}{}{%
\setlength{\topsep}{0pt}%
\setlength{\leftmargin}{0pt}%
\setlength{\rightmargin}{0pt}%
\setlength{\listparindent}{\parindent}%
\setlength{\itemindent}{\parindent}%
\setlength{\parsep}{0pt plus 1pt}%
\addtolength{\leftmargin}{#1}%
\addtolength{\rightmargin}{#2}%
}\item }{\end{list}}



\usepackage{pst-solides3d}
\usepackage{titletoc}
\titlecontents{chapter}[+3pc]
  {\addvspace{10pt}\sffamily\bfseries}
{\contentslabel[{\pscirclebox[fillstyle=solid,fillcolor=gray!25,
linecolor=gray!25,framesep=4pt]{\textcolor{white}{\thecontentslabel}}}]{2.5pc}}
  {}
  {\dotfill \normalfont\thecontentspage\ }

\titlecontents{section}[3pc]
  {\addvspace{2pt}\sffamily}
  {\contentslabel[\thecontentslabel]{1.8pc}}
  {}
  {\dotfill \normalfont\thecontentspage\ }

\titlecontents{subsection}[5pc]
  {\addvspace{2pt}\sffamily}
  {\contentslabel[\thecontentslabel]{1.8pc}}
  {}
  {\dotfill \normalfont\thecontentspage\ }

\titlecontents{subsubsection}[8pc]
  {\addvspace{2pt}\sffamily}
  {\contentslabel[\thecontentslabel]{3pc}}
  {}
  {\dotfill \normalfont\thecontentspage\ }
%{\;\titlerule\;\normalfont\thecontentspage\ }

\titlecontents{paragraph}[9pc]
  {\addvspace{2pt}\sffamily}
  {\contentslabel[\thecontentslabel]{3.5pc}}
  {}
  {\dotfill \normalfont\thecontentspage\ }


