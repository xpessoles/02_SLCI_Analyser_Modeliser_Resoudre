\documentclass[10pt]{article}
\input{style/coursHeadings}
\input{style/programHeadings}
\input{style/macros_SII}
\input{style/macros_Titres}
\input{style/macros_Frames}

%Si le boolen xp est vrai : compilation pour xabi
%Sinon compilation Damien
\newboolean{xp}
\setboolean{xp}{true}

\newboolean{prof}
\setboolean{prof}{true}

\newif\ifprof
%\proftrue
\proffalse

\newboolean{td}
\setboolean{td}{true}

\usepackage[%
    pdftitle={},
    pdfauthor={Xavier Pessoles},
    colorlinks=true,
    linkcolor=blue,
    citecolor=magenta]{hyperref}

\def\discipline{Sciences Industrielles de l'Ingénieur}

\def\xxtitre{\ifthenelse{\boolean{xp}}{CI 3 -- CIN : Étude du comportement cinématique des systèmes}{}}

\def\xxsoustitre{\ifthenelse{\boolean{xp}}{
Chapitre 4 -- Étude des chaînes fermées : Détermination des lois Entrée -- Sortie}{
}}


\def\xxauteur{\ifthenelse{\boolean{xp}}{
\noindent Xavier \textsc{Pessoles}}{
}}


\def\xxpied{\ifthenelse{\boolean{xp}}{
CI 4 : Cinématique \\
Ch 4 : Chaînes fermées -- Lois entrée -- sortie -- TD -- \ifthenelse{\boolean{prof}}{P}{E}%
}{
}}

\def\xxcathegorie{\ifthenelse{\boolean{xp}}{
2013 -- 2014 \\
Xavier \textsc{Pessoles}}{
Informatique - Cours}}

%---------------------------------------------------------------------------


\begin{document}

\ifthenelse{\boolean{xp}}{\input{style/enteteXP}}{\input{style/enteteDI}}



%\renewcommand{\baselinestretch}{1.2}
%\setlength{\parskip}{2ex plus 0.5ex minus 0.2ex}



\begin{comp}
\noindent \textbf{Résoudre :} à partir des modèles retenus :
%\begin{itemize}
%\item choisir une méthode de résolution analytique, graphique, numérique;
%\item mettre en \oe{}uvre une méthode de résolution.
%\end{itemize}

%\noindent \textit{Rés -- C1.1 :} Loi entrée sortie géométrique et cinématique -- Fermeture géométrique.

%\noindent \textit{Mod2 -- C4.1 :} Représentation par schéma bloc.
\end{comp}

\section*{Étude du téléphérique Vanoise Express}

\begin{flushright}
\textit{D'après concours E3A -- PSI -- 2014.}
\end{flushright}




\begin{minipage}[c]{.7\linewidth}


\begin{obj} 


\end{obj}


 
\end{minipage} \hfill
\begin{minipage}[c]{.25\linewidth}
\begin{center}
%\includegraphics[width=\textwidth]{images/Simulateur1}
\end{center}
\end{minipage}

\ifprof
\else
 On donne un extrait du cahier des charges.

\begin{minipage}[c]{.3\linewidth}
\begin{center}
%\includegraphics[width=\textwidth]{images/uc}

\textit{Diagramme des cas d'utilisation}
\end{center}
\end{minipage} \hfill
\begin{minipage}[c]{.3\linewidth}
\begin{center}
%\includegraphics[width=\textwidth]{images/ct}

\textit{Diagramme de contexte}
\end{center}
\end{minipage} \hfill
\begin{minipage}[c]{.38\linewidth}
\begin{center}
%\includegraphics[width=\textwidth]{images/req}

\textit{Diagramme partiel des exigences}
\end{center}
\end{minipage}


\begin{minipage}[c]{.33\linewidth}
\begin{center}
%\includegraphics[width=\textwidth]{images/Simulateur2}

\textit{Système réel}
\end{center}
\end{minipage} \hfill
\begin{minipage}[c]{.63\linewidth}
\begin{center}
%\includegraphics[width=\textwidth]{images/Simulateur3}

\textit{Modélisation plane}
\end{center}
\end{minipage}

\vspace{.25cm}



\fi
\subparagraph{}
\textit{Réaliser le graphe de liaisons de la structure articulée (système réel). }
\ifprof
\begin{corrige}
\begin{center}
%\includegraphics[width=.4\textwidth]{images/Corr_01}
\end{center}
\end{corrige}
\else
\fi


Les moteurs du téléphérique sont asservis en vitesse. La consigne est décrite sur la figure ci-dessous. L’immobilisation en gare est assurée par une consigne de vitesse nulle et par le frein à patin de service (frein à desserrage hydraulique).
 
Données :
\begin{itemize}
\item la distance totale à parcourir est $d_t=1830 \; \text{mètres}$;
\item entre l’ordre de départ et le départ, 6 secondes s’écoulent;
\item entre l’arrivée et l’ouverture des portes, 2 secondes s’écoulent;
\item le départ a lieu à l’instant $t=0\;s$, l’arrivée à l’instant $t_4$;
\item les instants $t_1$, $t_2$ et $t_3$ marquent les changements de consigne de vitesse.
\end{itemize}

\subparagraph{}
\textit{Pour cette question, on demande des résultats numériques avec 4 chiffres significatifs à exprimer en secondes ou mètres (unités SI).
\begin{enumerate}
\item Calculez numériquement $t_4-t_3$ pour respecter le critère de la fonction FT21 : « Distance à parcourir en petite vitesse avant l’arrivée en gare ».
\item Calculez numériquement $t_1$. En déduire numériquement la distance parcourue $d_a$ dans la phase d’accélération à $0,4\;m/s^2$.
\item Calculez numériquement $t_3-t_2$. En déduire numériquement la distance parcourue $d_d$ dans la phase de décélération à $-0,4\;m/s^2$.
\item Calculez numériquement $t_2-t_1$.
\item Calculez numériquement la durée totale $t_t$ de l’ordre de départ jusqu’à l’ouverture des portes. 
\end{enumerate}}

\subparagraph{}
\textit{Vérifiez le critère: « Durée d’un trajet (de l’ordre de départ jusqu’à l’ouverture des portes) » de la fonction FP1.}

En effet, afin de respecter les consignes de vitesse pour un trajet entre « Les Arcs » et « La Plagne », il est nécessaire que l’asservissement de vitesse des moteurs à courant continu ait des qualités en précision, stabilité et rapidité.

Documents : Voir annexe 1 pour certaines données (diamètre de la poulie motrice $D=4m$, rapport de réduction $k=1/20$...).

Modélisation des moteurs à courant continu

Notations :
\begin{itemize}
\item on notera $F(p)$ la transformée de Laplace d’une fonction du temps $f(t)$;
\item $u(t)$ : tension d’alimentation des moteurs;
\item $i(t)$ : intensité traversant un moteur;
\item $e(t)$ : force contre électromotrice d’un moteur;
\item $\omega_m(t)$ vitesse de rotation d’un moteur;
\item $c_m(t)$ : couple d’un seul moteur;
\item $c_r(t)$ : couple de perturbation engendré par le poids du téléphérique dans une pente et par l’action du vent, ramené sur l’axe des moteurs.
\end{itemize}

Hypothèses et données :
\begin{itemize}
\item on suppose les conditions initiales nulles;
\item les deux moteurs sont et fonctionnent de manière parfaitement identique.
\item $L=0.59 \; \text{mH}$ : inductance d’un moteur;
\item $R=0.0386 \; \Omega$ : résistance interne d’un moteur;
\item $f=6 \; N.m.s/rad	$ : coefficient de frottement visqueux équivalent ramené sur l’axe des moteurs;
\item $J=800 \; kg\cdot m^2$ : moment d’inertie total des pièces en rotation, ramené sur l’axe des moteurs;
•	 	avec   5.67 Nm/A   (constante de couple d’un moteur)
•	 	avec   =5.77 Vs/rad   (constante électrique d’un moteur)
•	Equations de la dynamique établie dans la partie précédente : 
	 



Dans ce qui suit, on désire respecter les critères suivants du cahier des charges partiel :





\end{document}

