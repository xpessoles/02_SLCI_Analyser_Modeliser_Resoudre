\setcounter{tocdepth}{2}
% \mtcselectlanguage{french} 


%  ------------------------------------------
% | Modification du formatage des sections : | 
%  ------------------------------------------

% Grands titres :
% ---------------

\newcommand{\titre}[1]{%
\begin{center}
      \bigskip
      \rule{\textwidth}{1pt}
      \par\vspace{0.1cm}
      
      \textbf{\large #1}
      \par\rule{\textwidth}{1pt}
    \end{center}
    \bigskip
  }

% Supprime le numéro du chapitre dans la numérotation des sections:
% -----------------------------------------------------------------
\makeatletter
\renewcommand{\thesection}{\@arabic\c@section}
\makeatother


% \titleformat{\chapter}[display]
% {\normalfont\Large\filcenter}
% {}
% {1pc}
% {\titlerule[1pt]
%   \vspace{1pc}%
%   \Huge}[\vspace{1ex}%
% \titlerule]


%%%% Chapitres Comme PY Pechard %%%%%%%%%
% numéro du chapitre
\DeclareFixedFont{\chapnumfont}{OT1}{phv}{b}{n}{80pt}
% pour le mot « Chapitre »
\DeclareFixedFont{\chapchapfont}{OT1}{phv}{m}{it}{40pt}
% pour le titre
\DeclareFixedFont{\chaptitfont}{T1}{phv}{b}{n}{25pt}

\definecolor{gris}{gray}{0.75}
\titleformat{\chapter}[display]%
	{\sffamily}%
	{\filleft\chapchapfont\color{gris}\chaptertitlename\
	\\
	\vspace{12pt}
	\chapnumfont\thechapter}%
	{16pt}%
	{\filleft\chaptitfont}%
	[\vspace{6pt}\titlerule\titlerule\titlerule]

%%%%  Fin Chapitres Comme PY Pechard %%%%%%%%%


% Section, subsection, subsubsection sans serifs :
% % ----------------------------------------------

% \makeatletter
% \renewcommand{\section}{\@startsection{section}{0}{0mm}%
% {\baselineskip}{.3\baselineskip}%
% {\normalfont\sffamily\Large\textbf}}%
% \makeatother

\makeatletter
\renewcommand{\@seccntformat}[1]{{\textcolor{bleu}{\csname
the#1\endcsname}\hspace{0.5em}}}
\makeatother

\makeatletter
\renewcommand{\section}{\@startsection{section}{1}{\z@}%
                       {-4ex \@plus -1ex \@minus -.4ex}%
                       {1ex \@plus.2ex }%
                       {\normalfont\Large\sffamily\bfseries}}%
\makeatother
 
\makeatletter
\renewcommand{\subsection}{\@startsection {subsection}{2}{\z@}
                          {-3ex \@plus -0.1ex \@minus -.4ex}%
                          {0.5ex \@plus.2ex }%
                          {\normalfont\large\sffamily\bfseries}}
\makeatother
 
\makeatletter
\renewcommand{\subsubsection}{\@startsection {subsubsection}{3}{\z@}
                          {-2ex \@plus -0.1ex \@minus -.2ex}%
                          {0.2ex \@plus.2ex }%
                          {\normalfont\large\sffamily\bfseries}}
\makeatother
 
\makeatletter             
\renewcommand{\paragraph}{\@startsection{paragraph}{4}{\z@}%
                                    {-2ex \@plus-.2ex \@minus .2ex}%
                                    {0.1ex}%               
{\normalfont\sffamily\bfseries}}
\makeatother
 
 
\makeatletter             
\renewcommand{\subparagraph}{\@startsection{subparagraph}{5}{\z@}%
                                    {-2ex \@plus-.2ex \@minus .2ex}%
                                    {0.1ex}%               
{\normalfont\bfseries Question }}
\makeatother
\renewcommand{\thesubparagraph}{\arabic{subparagraph}} 
\makeatletter

\setcounter{secnumdepth}{5}





% Formatage de la table des matières 
% Paquets nécessaires : titletoc ?

% Chapitre spéciaux écrits dans un nombre cerclé dans la table des matières.
\titlecontents{chapter}[+3pc]
  {\addvspace{10pt}\sffamily\bfseries}
{\contentslabel[{\pscirclebox[fillstyle=solid,fillcolor=gray!25,
linecolor=gray!25,framesep=4pt]{\textcolor{white}{\thecontentslabel}}}]{2.5pc}}
  {}
  {\dotfill \normalfont\thecontentspage\ }

\titlecontents{section}[3pc]
  {\addvspace{2pt}\sffamily}
  {\contentslabel[\thecontentslabel]{1.8pc}}
  {}
  {\dotfill \normalfont\thecontentspage\ }

\titlecontents{subsection}[5pc]
  {\addvspace{2pt}\sffamily}
  {\contentslabel[\thecontentslabel]{1.8pc}}
  {}
  {\dotfill \normalfont\thecontentspage\ }

\titlecontents{subsubsection}[8pc]
  {\addvspace{2pt}\sffamily}
  {\contentslabel[\thecontentslabel]{3pc}}
  {}
  {\dotfill \normalfont\thecontentspage\ }
%{\;\titlerule\;\normalfont\thecontentspage\ }

\titlecontents{paragraph}[9pc]
  {\addvspace{2pt}\sffamily}
  {\contentslabel[\thecontentslabel]{3.5pc}}
  {}
  {\dotfill \normalfont\thecontentspage\ }

%pour avoir l indentation dans minipage
\newdimen\oldparindent\oldparindent=\parindent

\makeatletter
\def\@iiiminipage#1#2[#3]#4{%
  \noindent
  \leavevmode
  \@pboxswfalse
  \setlength\@tempdima{#4}%
  \def\@mpargs{{#1}{#2}[#3]{#4}}%
  \setbox\@tempboxa\vbox\bgroup
    \color@begingroup
      \hsize\@tempdima
      \textwidth\hsize \columnwidth\hsize
      \@parboxrestore
      \parindent=\oldparindent
      \def\@mpfn{mpfootnote}\def\thempfn{\thempfootnote}\c@mpfootnote\z@
      \let\@footnotetext\@mpfootnotetext
      \let\@listdepth\@mplistdepth \@mplistdepth\z@
      \@minipagerestore
      \@setminipage}
\makeatother
