\documentclass[10pt,fleqn]{article} % Default font size and left-justified equations
\usepackage[%
    pdftitle={CIN : Cinématique du solide},
    pdfauthor={Xavier Pessoles}]{hyperref}
    
\input{style/new_style}
\input{style/macros_SII}

\usepackage{multicol}
\usepackage{style/schemabloc}
\fichetrue
%\fichefalse

\proftrue
%\proffalse

\tdtrue
%\tdfalse

%\courstrue
\coursfalse

\def\discipline{Sciences \\Industrielles de \\ l'Ingénieur}
\def\xxtete{Sciences Industrielles de l'Ingénieur}

\def\classe{PTSI}
\def\xxnumpartie{Partie 2}
\def\xxpartie{Découverte des Systèmes Linéaires Continus et Invariants\\
Analyse, Modélisation, Résolution}

\def\xxnumchapitre{Chapitre 2}
\def\xxchapitre{Modélisation des Systèmes Linéaires Continus Invariants -- Transformée de Laplace}

\def\xxtitreexo{Étude d'une cellule d'assemblage pour avion Falcon}
\def\xxsourceexo{\hspace{.2cm} D'après concours E3A -- PSI -- 2015.}


\def\xxposongletx{2}
\def\xxposonglettext{1.45}
\def\xxposonglety{20}
\def\xxonglet{Part. 2 -- Ch. 2}

\def\xxactivite{TD 2}
\def\xxauteur{\textsl{Xavier Pessoles}}

\def\xxcompetences{%
\textsl{%
\textbf{Savoirs et compétences :}\\
\noindent \textbf{Résoudre :} à partir des modèles retenus :
\begin{itemize}[label=\ding{112},font=\color{ocre}] 
\item ***
\item ***
\end{itemize}
\begin{itemize}[label=\ding{112},font=\color{ocre}] 
\item ***
\end{itemize}
%
%\noindent \textit{Mod2 -- C4.1 :} Représentation par schéma bloc.
}}

\def\xxfigures{
\includegraphics[width=\textwidth]{images/img_01}
}%figues de la page de garde

\def\xxpied{%
Partie 2 -- Découverte des SLCI -- Analyse, Modélisation, Résolution \\
Ch. 2 : Modélisation des Systèmes Linéaires Continus Invariants -- Transformée de Laplace -- \xxactivite%
}


\setcounter{secnumdepth}{5}
%---------------------------------------------------------------------------


\begin{document}
%\chapterimage{png/Fond_Cin}
\input{style/new_pagegarde}
\vspace{10cm}
\pagestyle{fancy}
\thispagestyle{plain}


\def\columnseprulecolor{\color{ocre}}
\setlength{\columnseprule}{0.4pt} 
\begin{multicols}{2}

Afin d'assembler les différents tronçons d'un avion Falcon 7X, la société Dassault utilise un Robot permettant les opérations suivantes :
\begin{enumerate}
\item mise en place des éléments à assembler;
\item perçage des éléments;
\item dépose d'un rivet;
\item pose d'une bague déformable;
\item serrage du rivet par déformation de la bague.
\end{enumerate}

\begin{obj} 

\end{obj}

Les équations caractéristiques du moteur à courant continu sont les suivantes : 
\begin{itemize}
\item $u(t)=e(t)+L\dfrac{\mathrm{d}i(t)}{\mathrm{d}t} + Ri(t)$
\item $e(t)=K_E \cdot \omega_m(t)$
\item $C_M(t)=K_C \cdot i(t)$;
\item $J_{eq}\dfrac{d\omega_m(t)}{dt} + f\omega_m(t) = C_M(t)-C_R(t)$.
\end{itemize}

Avec : 
\begin{itemize}
\item $u(t)$ : tension moteur;
\item $i(t)$ : courant moteur;
\item $e(t)$ : force contre-électromotrice;
\item $\omega_m(t)$ : fréquence de rotation moteur;
\item $C_M(t)$ : couple moteur;
\item $C_R(t)$ : couple résistant modélisant l'action de pesanteur.
\end{itemize}

\subparagraph{}
\textit{\`A partir des équations du moteur à courant continu, réaliser le schéma bloc du moteur à courant continu.}
\ifprof
\begin{corrige}
Les équations caractéristiques du moteur à courant continu dans le domaine de Laplace sont les suivantes : 
\begin{itemize}
\item $U(p)=E(p)+LpI(p) + RI(p)$
\item $E(p)=K_E \cdot \Omega_m(t)$
\item $C_M(p)=K_C \cdot I(p)$;
\item $J_{eq}p\Omega_m(p) + f\Omega_m(p) = C_M(p)-C_R(p)$.
\end{itemize}

\begin{center}
\noindent\footnotesize{
\begin{tikzpicture}
\sbEntree{E}

\sbComp{c1}{E}
    \sbRelier[$U(p)$]{E}{c1}
    
\sbBloc{b1}{$\dfrac{1}{R+Lp}$}{c1}
    \sbRelier{c1}{b1}

\sbBloc{b2}{$K_C$}{b1}
    \sbRelier[$I(p)$]{b1}{b2}

\sbComph{c2}{b2}
    \sbRelier[$C_M$]{b2}{c2}


\sbBloc{b3}{$\dfrac{1}{J_{eq} p + f}$}{c2}
    \sbRelier{c2}{b3}
    
\sbSortie{S}{b3}
    \sbRelier{b3}{S}
    \sbNomLien[0.8]{S}{$\Omega_m(p)$}
    
\sbDecaleNoeudy{b3}{n1}

\sbBlocr{r1}{$K_E$}{n1}
    \sbRelierxy[$E(p)$]{r1}{c1}
    \sbRelieryx{b3-S}{r1}

\sbDecaleNoeudy[-4]{c2}{n1}
\sbRelier[$C_R(p)$]{n1}{c2}
\end{tikzpicture}}

\end{center}
\end{corrige}
\else
\fi

\subparagraph{}
\textit{En considérant $C_R(p)=0$, déterminer la fonction de transfert $H_M(p)=\dfrac{\Omega_M(p)}{U(p)}$ sous sa forme canonique.}
\ifprof
\begin{corrige}
Si $C_R(p)=0$, on a : 
\begin{eqnarray*}
H_M(p) &=& \dfrac{K_C  \dfrac{1}{R+Lp}  \dfrac{1}{J_{eq}p + f} }{1+K_CK_E  \dfrac{1}{R+Lp}  \dfrac{1}{J_{eq}p + f}} \\
& = & \dfrac{K_C }{\left(R+Lp\right) \left(J_{eq}p + f \right) +K_CK_E} \\
& = & \dfrac{K_C }{ RJ_{eq}p +LJ_{eq}p^2 + fR +Lpf +K_CK_E} \\
& = & \dfrac{\dfrac{K_C}{K_CK_E + Rf} }{ \dfrac{LJ_{eq}}{K_CK_E + Rf}p^2 + \dfrac{ RJ_{eq}+Lf}{K_CK_E + Rf} p + 1} \\
\end{eqnarray*}
\end{corrige}
\else
\fi
\subparagraph{}
\textit{Montrer que $H_M(p)$ peut se mettre sous la forme simplifiée : $H_M(p)=\dfrac{K_C}{K_C K_E + RJ_{eq} p + LJ_{eq} p^2}$ puis sous la forme $H_M(p)=\dfrac{K_M}{\left(1+T_E p \right)\left(1+T_M p \right)}$ avec $T_E<T_M$.}
\ifprof
\begin{corrige}
En negligeant le frottement fluide, on a donc $f\simeq 0$. En conséquences, $H_M(p)=\dfrac{K_C}{K_C K_E + RJ_{eq} p + LJ_{eq} p^2}$. En mettant cette expressions sous forme canonique, on obtient  
$H_M(p)=\dfrac{1/K_E}{1+ \dfrac{RJ_{eq}}{K_C K_E} p + \dfrac{LJ_{eq}}{K_C K_E} p^2}$.

Développons l'expression donnée dans la question : 
$H_M(p)=\dfrac{K_M}{1+T_Ep + T_M p + T_E T_M p^2}$. En identifiant on a donc :

$$
\left\{ 
\begin{array}{l}
T_E + T_M = \dfrac{RJ_{eq}}{K_C K_E} \\
T_E \cdot T_M = \dfrac{LJ_{eq}}{K_C K_E} \\
\end{array}
\right.
$$


\end{corrige}
\else
\fi
\subparagraph{}
\textit{Quelle doit être la valeur de $K_G$ pour assurer un asservissement correct (c'est-à-dire que l'écart $\varepsilon$ doit être nul si la position de l'axe est identique à la consigne) ?}
\ifprof
\begin{corrige}
\end{corrige}
\else
\fi
\subparagraph{}
\textit{Compléter le schéma bloc de l'asservissement de l'axe du document réponse.}
\ifprof
\begin{corrige}
\end{corrige}
\else
\fi

\subparagraph{}
\textit{Mettre le schéma bloc sous la forme d'un schéma bloc à retour unitaire ayant la forme suivante en explicitant la fonction de transfert $H_C(p)$.}
\ifprof
\begin{corrige}
\end{corrige}
\else
\fi

On note $C(p)=1$.
\subparagraph{}
\textit{L'exigence *** est-elle vérifiée ?}
\ifprof
\begin{corrige}
\end{corrige}
\else
\fi

On note $C(p)=K_I \left( 1+\dfrac{A}{T_i p} \right) \left(1+ T_D p \right)$.
\subparagraph{}
\textit{L'exigence *** est-elle vérifiée ?}
\ifprof
\begin{corrige}
\end{corrige}
\else
\fi

Afin de vérifier maintenant le critère de rapidité, le document réponse donne la réponse temporelle de
l'axe à un échelon de position de $1 m$.

\subparagraph{}
\textit{L'exigence *** est-elle vérifiée ?}
\ifprof
\begin{corrige}
\end{corrige}
\else
\fi
\end{multicols}

\end{document}


